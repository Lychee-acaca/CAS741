\documentclass{article}

\usepackage{tabularx}
\usepackage{booktabs}
\usepackage{float}

\title{Reflection and Traceability Report on \progname}

\author{\authname}

\date{}

%% Comments

\usepackage{color}

\newif\ifcomments\commentstrue %displays comments
%\newif\ifcomments\commentsfalse %so that comments do not display

\ifcomments
\newcommand{\authornote}[3]{\textcolor{#1}{[#3 ---#2]}}
\newcommand{\todo}[1]{\textcolor{red}{[TODO: #1]}}
\else
\newcommand{\authornote}[3]{}
\newcommand{\todo}[1]{}
\fi

\newcommand{\wss}[1]{\authornote{blue}{SS}{#1}} 
\newcommand{\plt}[1]{\authornote{magenta}{TPLT}{#1}} %For explanation of the template
\newcommand{\an}[1]{\authornote{cyan}{Author}{#1}}

%% Common Parts

\newcommand{\progname}{ProgName} % PUT YOUR PROGRAM NAME HERE
\newcommand{\authname}{Team \#, Team Name
\\ Student 1 name
\\ Student 2 name
\\ Student 3 name
\\ Student 4 name} % AUTHOR NAMES                  

\usepackage{hyperref}
    \hypersetup{colorlinks=true, linkcolor=blue, citecolor=blue, filecolor=blue,
                urlcolor=blue, unicode=false}
    \urlstyle{same}
                                


\begin{document}

\maketitle

\section{Changes in Response to Feedback}

The following subsections organize the feedback received throughout the project
according to the corresponding deliverables: the Software Requirements
Specification (SRS) and Hazard Analysis, the Design and Design Documentation,
and the Verification and Validation (VnV) Plan and Report.

\subsection{SRS and Hazard Analysis}

This section summarizes the modifications made to the SRS document in response
to the feedback received. The revisions primarily focused on improving document
structure and enhancing content clarity, as recommended by Dr. Smith in issue
\href{https://github.com/Lychee-acaca/CAS741/issues/13}{\#13}.

As shown in Table \ref{tab:SRSChanges}, the following changes were made in
response to the feedback received.

\begin{table}[ht] 
\centering
\begin{tabular}{|p{8cm}|p{2.5cm}|}
\hline
\textbf{Feedback Description} & \textbf{PR Number} \\
\hline
Refine the description of intended readers to better reflect their academic and
technical background. &
\href{https://github.com/Lychee-acaca/CAS741/pull/35}{\#35} \\
\hline
Clarify the scope of requirements by explicitly defining what is included and
excluded from the project. &
\href{https://github.com/Lychee-acaca/CAS741/pull/35}{\#35} \\
\hline
Improve clarity in the description of system constraints, specifically the
requirement for low-performance embedded devices. &
\href{https://github.com/Lychee-acaca/CAS741/pull/35}{\#35} \\
\hline
Clarify the terminology and definitions, such as for QRS complex and R-wave. &
\href{https://github.com/Lychee-acaca/CAS741/pull/35}{\#35} \\
\hline
Refactor functional and nonfunctional requirements to ensure consistency and
clarity. & \href{https://github.com/Lychee-acaca/CAS741/pull/35}{\#35} \\
\hline
\end{tabular}
\caption{Summary of Changes for SRS and Hazard Analysis}
\label{tab:SRSChanges}
\end{table}

\subsection{Design and Design Documentation}

This section summarizes the modifications made to the MG and MIS document in
response to the feedback received. The revisions primarily focused on improving
the clarity of the module hierarchy and enhancing the explanation of each
module's functionality, as Dr. Smith recommended in issue
\href{https://github.com/Lychee-acaca/CAS741/issues/21}{\#21}. In addition,
Domain Expert Baptiste raised an issue related to document hyperlinks in
\href{https://github.com/Lychee-acaca/CAS741/issues/25}{\#25}., which made the
document easier to read.

As shown in Table \ref{tab:MG}, the following changes were made in response to
the feedback received according to the MG Document.

\begin{table}[ht] \centering \begin{tabular}{|p{8cm}|p{2.5cm}|} \hline
\textbf{Feedback Description} & \textbf{PR Number} \\
\hline Refactor the module hierarchy to improve clarity and structure. &
\href{https://github.com/Lychee-acaca/CAS741/pull/35}{\#35} \\
\hline Clarify the services provided by the different modules, especially the
Rwave Detect and Math modules. &
\href{https://github.com/Lychee-acaca/CAS741/pull/35}{\#35} \\
\hline
Improve the description of implementation methods and algorithms used in the
Pan-Tompkins Algorithm and Specified IIR Filter modules. &
\href{https://github.com/Lychee-acaca/CAS741/pull/35}{\#35} \\
\hline Refine the explanations of the Input Output Processing and Error Handler
modules. & \href{https://github.com/Lychee-acaca/CAS741/pull/35}{\#35} \\
\hline
\end{tabular} \caption{Summary of Changes for MG Document} \label{tab:MG}
\end{table}

As shown in Table \ref{tab:MIS}, the following changes were made in response to
the feedback received according to the MIS Document.

\begin{table}[ht] 
\centering 
\begin{tabular}{|p{8cm}|p{2.5cm}|} 
\hline
\textbf{Feedback Description} & \textbf{PR Number} \\
\hline 
Add previously missing Environment Variables for certain modules. &
\href{https://github.com/Lychee-acaca/CAS741/pull/37}{\#37} \\
\hline 
Correct the transition semantics in the Access Routine section. &
\href{https://github.com/Lychee-acaca/CAS741/pull/37}{\#37} \\
\hline 
Add several hyperlinks to improve navigation between modules. &
\href{https://github.com/Lychee-acaca/CAS741/pull/26}{\#26} \\
\hline 
\end{tabular} 
\caption{Summary of Changes for MIS Document} 
\label{tab:MIS}
\end{table}

\subsection{VnV Plan and Report}

\section{Challenge Level and Extras}

\subsection{Challenge Level}

\plt{State the challenge level (advanced, general, basic) for your project.  Your challenge level should exactly match what is included in your problem statement.  This should be the challenge level agreed on between you and the course instructor.}

\subsection{Extras}

\plt{Summarize the extras (if any) that were tackled by this project.  Extras
can include usability testing, code walkthroughs, user documentation, formal
proof, GenderMag personas, Design Thinking, etc.  Extras should have already
been approved by the course instructor as included in your problem statement.}

\section{Design Iteration (LO11 (PrototypeIterate))}

The final design of the R-peak detection system retained the core structure of
the Pan-Tompkins algorithm without modifications to its detection logic. The
primary design iteration focused on handling abnormal or invalid input cases
more effectively. During early testing, issues were identified when the system
encountered empty, non-numeric, or improperly formatted ECG data, which could
lead to runtime errors or undefined behavior. To address this, input validation
and exception handling mechanisms were introduced, ensuring that the system
could safely detect and report invalid inputs without interrupting execution.
These adjustments improved the robustness and user safety of the application,
aligning the implementation with practical usage scenarios where unexpected or
corrupted input data might occur.

\section{Design Decisions (LO12)}

\plt{Reflect and justify your design decisions.  How did limitations,
 assumptions, and constraints influence your decisions?  Discuss each of these
 separately.}

\section{Economic Considerations (LO23)}

This project is a reimplementation of the Pan-Tompkins algorithm for R-peak
detection in ECG signals and is intended as an open-source tool rather than a
commercial product. There are no associated production costs or pricing
considerations. The project will be shared on a public repository platform with
clear documentation and example usage to attract developers, students, and
researchers in biomedical signal processing. Its value lies in providing a
reliable, accessible, and well-documented implementation for educational use and
integration into other health-related applications.

\section{Reflection on Project Management (LO24)}

\subsection{How Does Your Project Management Compare to Your Development Plan}

The project closely followed the initial development plan. The technologies
outlined in the plan, such as Docker for containerization, Google Test for unit
testing, GitHub Actions for continuous integration, \texttt{cpplint} for code
style enforcement, and \texttt{cppcheck} for static code analysis, were all
implemented as intended. These tools facilitated a streamlined development
process and ensured code quality and consistency.

\subsection{What Went Well?}

Several aspects of the project management were successful:

\begin{itemize}
    \item \textbf{Tool Integration:} The integration of Docker, Google Test,
    GitHub Actions, \texttt{cpplint}, and \texttt{cppcheck} into the workflow
    enhanced development efficiency and code reliability.
    \item \textbf{Continuous Integration:} GitHub Actions provided automated
    testing and analysis, allowing for immediate feedback on code changes and
    reducing the likelihood of introducing errors.
\end{itemize}

\subsection{What Went Wrong?}

One challenge encountered was with the implementation of Git pre-commit hooks.
Initially, pre-commit hooks were planned to automatically run tests before each
commit. However, issues related to file read/write permissions prevented certain
tests, especially those involving file operations, from executing correctly in
the pre-commit context. Since the continuous integration pipeline via GitHub
Actions already performed comprehensive testing, maintaining the pre-commit
hooks was deemed redundant and subsequently removed from the workflow.

\subsection{What Would You Do Differently Next Time?}

In future projects, the following adjustments would be made:

\begin{itemize}
    \item \textbf{Evaluate Tool Redundancy:} Assess the necessity and potential
    overlap of tools like pre-commit hooks and continuous integration pipelines
    to avoid redundant efforts.
    \item \textbf{Focus on Effective Tools:} Prioritize tools that offer the
    most value and align with the workflow, ensuring that each tool serves a
    distinct and necessary purpose.
    \item \textbf{Early Testing of Tools:} Conduct thorough testing of all
    planned tools in the early stages of the project to identify and address any
    compatibility or functionality issues.
\end{itemize}

\section{Reflection on Capstone}

This section summarizes the relevant academic courses and additional technical
skills that contributed to the successful completion of the R-wave detection
project.

\subsection{Which Courses Were Relevant}

The capstone project for R-peak detection primarily drew on knowledge from
several relevant courses. The \textit{C++ Programming} course provided essential
skills for system implementation, including memory management, modular
programming, and testing frameworks. Additionally, the \textit{Digital Signal
Processing} and \textit{Signals and Systems} courses were directly applicable,
as they covered the theoretical foundation of ECG signal processing, filtering
techniques, and time-domain analysis, which are critical components of the
Pan-Tompkins algorithm.

\subsection{Knowledge/Skills Outside of Courses}

Beyond coursework, several additional skills were required for this project.
Experience with continuous integration tools, such as GitHub Actions, was
necessary to establish automated testing and analysis pipelines. Knowledge of
Docker was also acquired to manage a consistent development environment.
Furthermore, familiarity with code analysis tools like \texttt{cpplint} and
\texttt{cppcheck} was developed to ensure code quality and compliance with C++
standards. These tools and practices, although not covered in previous courses,
were essential for maintaining a reliable and maintainable codebase in a modern
software development workflow.

\end{document}