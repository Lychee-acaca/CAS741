\documentclass{article}

\usepackage{tabularx}
\usepackage{booktabs}
\usepackage{float}

\title{Reflection and Traceability Report on \progname}

\author{\authname}

\date{}

%% Comments

\usepackage{color}

\newif\ifcomments\commentstrue %displays comments
%\newif\ifcomments\commentsfalse %so that comments do not display

\ifcomments
\newcommand{\authornote}[3]{\textcolor{#1}{[#3 ---#2]}}
\newcommand{\todo}[1]{\textcolor{red}{[TODO: #1]}}
\else
\newcommand{\authornote}[3]{}
\newcommand{\todo}[1]{}
\fi

\newcommand{\wss}[1]{\authornote{blue}{SS}{#1}} 
\newcommand{\plt}[1]{\authornote{magenta}{TPLT}{#1}} %For explanation of the template
\newcommand{\an}[1]{\authornote{cyan}{Author}{#1}}

%% Common Parts

\newcommand{\progname}{ProgName} % PUT YOUR PROGRAM NAME HERE
\newcommand{\authname}{Team \#, Team Name
\\ Student 1 name
\\ Student 2 name
\\ Student 3 name
\\ Student 4 name} % AUTHOR NAMES                  

\usepackage{hyperref}
    \hypersetup{colorlinks=true, linkcolor=blue, citecolor=blue, filecolor=blue,
                urlcolor=blue, unicode=false}
    \urlstyle{same}
                                


\begin{document}

\maketitle

\section{Changes in Response to Feedback}

The following subsections organize the feedback received throughout the project
according to the corresponding deliverables: the Software Requirements
Specification (SRS) and Hazard Analysis, the Design and Design Documentation,
and the Verification and Validation (VnV) Plan and Report.

\subsection{SRS and Hazard Analysis}

This section summarizes the modifications made to the SRS document in response
to the feedback received. The revisions primarily focused on improving document
structure and enhancing content clarity, as recommended by Dr. Smith in issue
\href{https://github.com/Lychee-acaca/CAS741/issues/13}{\#13}.

As shown in Table \ref{tab:SRSChanges}, the following changes were made in
response to the feedback received.

\begin{table}[ht] 
\centering
\begin{tabular}{|p{8cm}|p{2.5cm}|}
\hline
\textbf{Feedback Description} & \textbf{PR Number} \\
\hline
Refine the description of intended readers to better reflect their academic and
technical background. &
\href{https://github.com/Lychee-acaca/CAS741/pull/35}{\#35} \\
\hline
Clarify the scope of requirements by explicitly defining what is included and
excluded from the project. &
\href{https://github.com/Lychee-acaca/CAS741/pull/35}{\#35} \\
\hline
Improve clarity in the description of system constraints, specifically the
requirement for low-performance embedded devices. &
\href{https://github.com/Lychee-acaca/CAS741/pull/35}{\#35} \\
\hline
Clarify the terminology and definitions, such as for QRS complex and R-wave. &
\href{https://github.com/Lychee-acaca/CAS741/pull/35}{\#35} \\
\hline
Refactor functional and nonfunctional requirements to ensure consistency and
clarity. & \href{https://github.com/Lychee-acaca/CAS741/pull/35}{\#35} \\
\hline
\end{tabular}
\caption{Summary of Changes for SRS and Hazard Analysis}
\label{tab:SRSChanges}
\end{table}

\subsection{Design and Design Documentation}

This section summarizes the modifications made to the MG and MIS document in
response to the feedback received. The revisions primarily focused on improving
the clarity of the module hierarchy and enhancing the explanation of each
module's functionality, as Dr. Smith recommended in issue
\href{https://github.com/Lychee-acaca/CAS741/issues/21}{\#21}. In addition,
Domain Expert Baptiste raised an issue related to document hyperlinks in
\href{https://github.com/Lychee-acaca/CAS741/issues/25}{\#25}., which made the
document easier to read.

As shown in Table \ref{tab:MG}, the following changes were made in response to
the feedback received according to the MG Document.

\begin{table}[ht] \centering \begin{tabular}{|p{8cm}|p{2.5cm}|} \hline
\textbf{Feedback Description} & \textbf{PR Number} \\
\hline Refactor the module hierarchy to improve clarity and structure. &
\href{https://github.com/Lychee-acaca/CAS741/pull/35}{\#35} \\
\hline Clarify the services provided by the different modules, especially the
Rwave Detect and Math modules. &
\href{https://github.com/Lychee-acaca/CAS741/pull/35}{\#35} \\
\hline
Improve the description of implementation methods and algorithms used in the
Pan-Tompkins Algorithm and Specified IIR Filter modules. &
\href{https://github.com/Lychee-acaca/CAS741/pull/35}{\#35} \\
\hline Refine the explanations of the Input Output Processing and Error Handler
modules. & \href{https://github.com/Lychee-acaca/CAS741/pull/35}{\#35} \\
\hline
\end{tabular} \caption{Summary of Changes for MG Document} \label{tab:MG}
\end{table}

As shown in Table \ref{tab:MIS}, the following changes were made in response to
the feedback received according to the MIS Document.

\begin{table}[ht] 
\centering 
\begin{tabular}{|p{8cm}|p{2.5cm}|} 
\hline
\textbf{Feedback Description} & \textbf{PR Number} \\
\hline 
Add previously missing Environment Variables for certain modules. &
\href{https://github.com/Lychee-acaca/CAS741/pull/37}{\#37} \\
\hline 
Correct the transition semantics in the Access Routine section. &
\href{https://github.com/Lychee-acaca/CAS741/pull/37}{\#37} \\
\hline 
Add several hyperlinks to improve navigation between modules. &
\href{https://github.com/Lychee-acaca/CAS741/pull/26}{\#26} \\
\hline 
\end{tabular} 
\caption{Summary of Changes for MIS Document} 
\label{tab:MIS}
\end{table}

\subsection{VnV Plan and Report}

\section{Challenge Level and Extras}

\subsection{Challenge Level}

\plt{State the challenge level (advanced, general, basic) for your project.  Your challenge level should exactly match what is included in your problem statement.  This should be the challenge level agreed on between you and the course instructor.}

\subsection{Extras}

\plt{Summarize the extras (if any) that were tackled by this project.  Extras
can include usability testing, code walkthroughs, user documentation, formal
proof, GenderMag personas, Design Thinking, etc.  Extras should have already
been approved by the course instructor as included in your problem statement.}

\section{Design Iteration (LO11 (PrototypeIterate))}

\plt{Explain how you arrived at your final design and implementation.  How did
the design evolve from the first version to the final version?} 

\plt{Don't just say what you changed, say why you changed it.  The needs of the
client should be part of the explanation.  For example, if you made changes in
response to usability testing, explain what the testing found and what changes
it led to.}

\section{Design Decisions (LO12)}

\plt{Reflect and justify your design decisions.  How did limitations,
 assumptions, and constraints influence your decisions?  Discuss each of these
 separately.}

\section{Economic Considerations (LO23)}

\plt{Is there a market for your product? What would be involved in marketing
your product? What is your estimate of the cost to produce a version that you
could sell?  What would you charge for your product?  How many units would you
have to sell to make money? If your product isn't something that would be sold,
like an open source project, how would you go about attracting users?  How many
potential users currently exist?}

\section{Reflection on Project Management (LO24)}

\plt{This question focuses on processes and tools used for project management.}

\subsection{How Does Your Project Management Compare to Your Development Plan}

\plt{Did you follow your Development plan, with respect to the team meeting
plan, team communication plan, team member roles and workflow plan.  Did you use
the technology you planned on using?}

\subsection{What Went Well?}

\plt{What went well for your project management in terms of processes and
technology?}

\subsection{What Went Wrong?}

\plt{What went wrong in terms of processes and technology?}

\subsection{What Would you Do Differently Next Time?}

\plt{What will you do differently for your next project?}

\section{Reflection on Capstone}

\plt{This question focuses on what you learned during the course of the capstone project.}

\subsection{Which Courses Were Relevant}

\plt{Which of the courses you have taken were relevant for the capstone project?}

\subsection{Knowledge/Skills Outside of Courses}

\plt{What skills/knowledge did you need to acquire for your capstone project
that was outside of the courses you took?}

\end{document}