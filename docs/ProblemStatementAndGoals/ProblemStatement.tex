\documentclass{article}

\usepackage{tabularx}
\usepackage{booktabs}

\title{Problem Statement and Goals\\\progname}

\author{\authname}

\date{January 16, 2025}

%% Comments

\usepackage{color}

\newif\ifcomments\commentstrue %displays comments
%\newif\ifcomments\commentsfalse %so that comments do not display

\ifcomments
\newcommand{\authornote}[3]{\textcolor{#1}{[#3 ---#2]}}
\newcommand{\todo}[1]{\textcolor{red}{[TODO: #1]}}
\else
\newcommand{\authornote}[3]{}
\newcommand{\todo}[1]{}
\fi

\newcommand{\wss}[1]{\authornote{blue}{SS}{#1}} 
\newcommand{\plt}[1]{\authornote{magenta}{TPLT}{#1}} %For explanation of the template
\newcommand{\an}[1]{\authornote{cyan}{Author}{#1}}

%% Common Parts

\newcommand{\progname}{ProgName} % PUT YOUR PROGRAM NAME HERE
\newcommand{\authname}{Team \#, Team Name
\\ Student 1 name
\\ Student 2 name
\\ Student 3 name
\\ Student 4 name} % AUTHOR NAMES                  

\usepackage{hyperref}
    \hypersetup{colorlinks=true, linkcolor=blue, citecolor=blue, filecolor=blue,
                urlcolor=blue, unicode=false}
    \urlstyle{same}
                                


\begin{document}

\maketitle

\begin{table}[hp]
\caption{Revision History} \label{TblRevisionHistory}
\begin{tabularx}{\textwidth}{llX}
\toprule
\textbf{Date} & \textbf{Developer(s)} & \textbf{Change}\\
\midrule
January 16, 2025 & Junwei Lin & Creation\\
January 22, 2025 & Junwei Lin & Dr. Smith's feedback\\
\bottomrule
\end{tabularx}
\end{table}

\section{Problem Statement}

R-wave detection is a critical task in Electrocardiogram (ECG) signal processing, serving important purposes such as heart rate calculation, arrhythmia analysis, cardiac conduction evaluation, and heart disease diagnosis. The patient's ECG signal is sampled by the sensor and converted into a digital signal, where we can detect R-waves. However, since the sampling is not under an ideal condition, a large amount of clutter wave is mixed into the original signal, such as electrical noises and utility frequency.

\subsection{Problem}

A strong filtering algorithm is needed to filter out redundant signals, and then accurately capture the index of R-wave peaks.

\subsection{Inputs and Outputs}

\subsubsection{Inputs}

\begin{enumerate}
    \item Sampling data file, comes from \href{https://physionet.org/content/mitdb/1.0.0/} {MIT-BIH Arrhythmia Database}
    \item Sampling frequency of the data file
    \item (Optional) Annotated data from cardiologists, used to evaluate program detecting accuracy
\end{enumerate}

\subsubsection{Outputs}

\begin{enumerate}
    \item Index of R-wave peaks
    \item A graph that shows all the R-wave peaks
    \item (Optional) Root Mean Square Error (RMSE) between each detected R-wave peak time and annotated time from cardiologists
\end{enumerate}

\subsection{Stakeholders}

The stakeholders include but are not limited to patients, doctors, cardiologists, nurses, technicians, medical device manufacturers, researchers, all of whom rely on accurate, reliable, and efficient ECG signal processing for diagnosis, treatment, innovation, and patient care.

\subsection{Environment}

Windows, Linux, MacOS, and most light-weight embedding systems are able to run this program.

\section{Goals}

\begin{enumerate}
    \item Given a single-channel unfiltered ECG signal, find the index of each R-wave peak.
    \item Given correct annotated data, calculate the RMSE between each detected R-wave peak time and annotated time.
\end{enumerate}

\section{Stretch Goals}

Implements a variety of R-wave detection algorithms for users to choose from.

\section{Similar libraries}

The \href{https://docs.scipy.org/doc/scipy-1.15.0/reference/signal.html} {signal processing module of SciPy} can calculate filter parameters like this program partly does, so it can be used as a pseudo-oracles for later testing.

\section{Challenge Level and Extras}

This is a problem with a general-level challenge. It is not a research project, but it still requires knowledge of IIR and FIR filter design. Additionally, solving this problem requires understanding some algorithms belonging to this field. Doxygen will be widely used to generate the required documentation, which could be a potential extra.

\end{document}
