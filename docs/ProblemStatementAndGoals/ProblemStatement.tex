\documentclass{article}

\usepackage{tabularx}
\usepackage{booktabs}

\title{Problem Statement and Goals\\\progname}

\author{\authname}

\date{January 16, 2025}

%% Comments

\usepackage{color}

\newif\ifcomments\commentstrue %displays comments
%\newif\ifcomments\commentsfalse %so that comments do not display

\ifcomments
\newcommand{\authornote}[3]{\textcolor{#1}{[#3 ---#2]}}
\newcommand{\todo}[1]{\textcolor{red}{[TODO: #1]}}
\else
\newcommand{\authornote}[3]{}
\newcommand{\todo}[1]{}
\fi

\newcommand{\wss}[1]{\authornote{blue}{SS}{#1}} 
\newcommand{\plt}[1]{\authornote{magenta}{TPLT}{#1}} %For explanation of the template
\newcommand{\an}[1]{\authornote{cyan}{Author}{#1}}

%% Common Parts

\newcommand{\progname}{ProgName} % PUT YOUR PROGRAM NAME HERE
\newcommand{\authname}{Team \#, Team Name
\\ Student 1 name
\\ Student 2 name
\\ Student 3 name
\\ Student 4 name} % AUTHOR NAMES                  

\usepackage{hyperref}
    \hypersetup{colorlinks=true, linkcolor=blue, citecolor=blue, filecolor=blue,
                urlcolor=blue, unicode=false}
    \urlstyle{same}
                                


\begin{document}

\maketitle

\begin{table}[hp]
\caption{Revision History} \label{TblRevisionHistory}
\begin{tabularx}{\textwidth}{llX}
\toprule
\textbf{Date} & \textbf{Developer(s)} & \textbf{Change}\\
\midrule
January 16, 2025 & Junwei Lin & Creation\\
\bottomrule
\end{tabularx}
\end{table}

\section{Problem Statement}

R-wave detection is a critical task in ECG (electrocardiogram) signal processing, serving important purposes such as heart rate calculation, arrhythmia analysis, cardiac conduction evaluation, and heart disease diagnosis. The patient's ECG signal is sampled by the sensor and converted into a digital signal, where we can detect R-waves. However, since the sampling is not under an ideal condition, a large amount of clutter wave is mixed into the original signal, such as utility frequency. So we need a strong filtering algorithm to filter out redundant signals, and then accurately capture the peaks of R-waves.

\subsection{Problem}

\subsection{Inputs and Outputs}

\subsubsection{Inputs}

\begin{enumerate}
    \item Sampling frequency
    \item Sampling data file
\end{enumerate}

\subsubsection{Outputs}

\begin{enumerate}
    \item Index of R-wave peaks
    \item A graph that shows all the R-wave peaks
\end{enumerate}

\subsection{Stakeholders}

The stakeholders include but are not limited to patients, doctors, cardiologists, nurses, technicians, medical device manufacturers, researchers, all of whom rely on accurate, reliable, and efficient ECG signal processing for diagnosis, treatment, innovation, and patient care.

\subsection{Environment}

debian-12.9.0-amd64

\section{Goals}

Given a single channel unfiltered ECG signal, find the index of each R wave peak.

\section{Stretch Goals}

\section{Challenge Level and Extras}

\wss{State your expected challenge level (advanced, general or basic).  The
challenge can come through the required domain knowledge, the implementation or
something else.  Usually the greater the novelty of a project the greater its
challenge level.  You should include your rationale for the selected level.
Approval of the level will be part of the discussion with the instructor for
approving the project.  The challenge level, with the approval (or request) of
the instructor, can be modified over the course of the term.}

\wss{Teams may wish to include extras as either potential bonus grades, or to
make up for a less advanced challenge level.  Potential extras include usability
testing, code walkthroughs, user documentation, formal proof, GenderMag
personas, Design Thinking, etc.  Normally the maximum number of extras will be
two.  Approval of the extras will be part of the discussion with the instructor
for approving the project.  The extras, with the approval (or request) of the
instructor, can be modified over the course of the term.}

\end{document}
