\documentclass[12pt, titlepage]{article}

\usepackage{amsmath, mathtools}

\usepackage[round]{natbib}
\usepackage{amsfonts}
\usepackage{amssymb}
\usepackage{graphicx}
\usepackage{colortbl}
\usepackage{xr}
\usepackage{hyperref}
\usepackage{longtable}
\usepackage{xfrac}
\usepackage{tabularx}
\usepackage{float}
\usepackage{siunitx}
\usepackage{booktabs}
\usepackage{multirow}
\usepackage[section]{placeins}
\usepackage{caption}
\usepackage{fullpage}

\hypersetup{ bookmarks=true,     % show bookmarks bar?
colorlinks=true,       % false: boxed links; true: colored links
linkcolor=red,          % color of internal links (change box color with
                        % linkbordercolor)
citecolor=blue,      % color of links to bibliography
filecolor=magenta,  % color of file links
urlcolor=cyan          % color of external links
}

\usepackage{array}

\externaldocument{../../SRS/SRS}

%% Comments

\usepackage{color}

\newif\ifcomments\commentstrue %displays comments
%\newif\ifcomments\commentsfalse %so that comments do not display

\ifcomments
\newcommand{\authornote}[3]{\textcolor{#1}{[#3 ---#2]}}
\newcommand{\todo}[1]{\textcolor{red}{[TODO: #1]}}
\else
\newcommand{\authornote}[3]{}
\newcommand{\todo}[1]{}
\fi

\newcommand{\wss}[1]{\authornote{blue}{SS}{#1}} 
\newcommand{\plt}[1]{\authornote{magenta}{TPLT}{#1}} %For explanation of the template
\newcommand{\an}[1]{\authornote{cyan}{Author}{#1}}

%% Common Parts

\newcommand{\progname}{ProgName} % PUT YOUR PROGRAM NAME HERE
\newcommand{\authname}{Team \#, Team Name
\\ Student 1 name
\\ Student 2 name
\\ Student 3 name
\\ Student 4 name} % AUTHOR NAMES                  

\usepackage{hyperref}
    \hypersetup{colorlinks=true, linkcolor=blue, citecolor=blue, filecolor=blue,
                urlcolor=blue, unicode=false}
    \urlstyle{same}
                                


\begin{document}

\title{Module Interface Specification for \progname{}}

\author{\authname}

\date{\today}

\maketitle

\pagenumbering{roman}

\section{Revision History}

\begin{tabularx}{\textwidth}{p{3cm}p{2cm}X} \toprule {\bf Date} & {\bf Version}
& {\bf Notes}\\
\midrule
Mar 17, 2025 & 1.0 & Creation\\
April 18, 2025 & 1.1 & Modify the document according to the received feedback\\
\bottomrule
\end{tabularx}

~\newpage

\section{Symbols, Abbreviations and Acronyms}

See SRS Documentation at
\url{https://github.com/Lychee-acaca/CAS741/blob/main/docs/SRS/SRS.pdf}.

%\wss{Also add any additional symbols, abbreviations or acronyms}

\newpage

\tableofcontents

\newpage

\pagenumbering{arabic}

\section{Introduction}

The following document details the Module Interface Specifications for
\progname.

This project focuses on reimplementing the Pan-Tompkins algorithm for R-wave
detection in ECG signal processing.

Complementary documents include the System Requirement Specifications and Module
Guide.  The full documentation and implementation can be found at
\url{https://github.com/Lychee-acaca/CAS741}.

\section{Notation}

The structure of the MIS for modules comes from \citet{HoffmanAndStrooper1995},
with the addition that template modules have been adapted from
\cite{GhezziEtAl2003}.  The mathematical notation comes from Chapter 3 of
\citet{HoffmanAndStrooper1995}.  For instance, the symbol := is used for a
multiple assignment statement and conditional rules follow the form $(c_1
\Rightarrow r_1 | c_2 \Rightarrow r_2 | ... | c_n \Rightarrow r_n )$.

The following table summarizes the primitive data types used by \progname. 

\begin{center}
\renewcommand{\arraystretch}{1.2}
\noindent 
\begin{tabular}{l l p{7.5cm}} 
\toprule 
\textbf{Data Type} & \textbf{Notation} & \textbf{Description}\\ 
\midrule
character & $\text{char}$ & a single symbol or digit\\
integer & $\mathbb{Z}$ & a number without a fractional component in (-$\infty$,
$\infty$) \\
natural number & $\mathbb{N}$ & a number without a fractional component in [1,
$\infty$) \\
real & $\mathbb{R}$ & any number in (-$\infty$, $\infty$)\\
\bottomrule
\end{tabular}
\end{center}

\noindent
The specification of \progname \ uses some derived data types: sequences and
strings. Sequences are lists filled with elements of the same data type, they
are represented using a type with a superscript, for example, $\mathbb{R}^n$
represents a real number sequence of length n. Strings are sequences of
characters. In addition, \progname \ uses functions, which are defined by the
data types of their inputs and outputs. Local functions are described by giving
their type signature followed by their specification.

\section{Module Decomposition}

The following table is taken directly from the Module Guide document for this
project.

\begin{table}[h!]
\centering
\begin{tabular}{p{0.3\textwidth} p{0.6\textwidth}}
\toprule
\textbf{Level 1} & \textbf{Level 2}\\
\midrule

{Hardware-Hiding Module} & ~ \\
\midrule

\multirow{5}{0.3\textwidth}{Behaviour-Hiding Module}
& General Digital Filter Module \\
& Specified IIR Filter Module \\
& Math Module \\
& Pan Tompkins Algorithm Module \\
& Annotated Data RMSE Module \\
\midrule

\multirow{4}{0.3\textwidth}{Software Decision Module}
& Rwave Detect Module \\
& Input Output Processing Module \\
& Error Handler Module \\
& Logger Module \\
\bottomrule

\end{tabular}
\caption{Module Hierarchy}
\label{TblMH}
\end{table}

\newpage
~\newpage

\section{MIS of Hardware-Hiding Module} \label{MIS_HH}

This section outlines the formal specification of the Hardware-Hiding Module,
which provides an abstraction layer for interacting with hardware devices.

\subsection{Module}

Hardware-Hiding

\subsection{Uses}

None

\subsection{Syntax}

This section describes the syntax used in the Hardware-Hiding Module, including
the constants, programs, and routines that can be accessed.

\subsubsection{Exported Constants}

None

\subsubsection{Exported Access Programs}

\begin{center}
\begin{tabular}{p{2cm} p{4cm} p{4cm} p{2cm}}
\hline
\textbf{Name} & \textbf{In} & \textbf{Out} & \textbf{Exceptions} \\
\hline
open & $\text{string}$, $\mathbb{N}$, $\mathbb{N}$ & $\mathbb{Z}$ & - \\
write & $\mathbb{Z}$, $\mathbb{N}^n$, $\mathbb{N}$ & $\mathbb{N}$ & - \\
read & $\mathbb{Z}$, $\mathbb{N}^n$, $\mathbb{N}$ & $\mathbb{N}$ & - \\
\hline
\end{tabular}
\end{center}

\subsection{Semantics}

This section provides the detailed functionality and behavior of the module’s
functions.

\subsubsection{State Variables}

None

\subsubsection{Environment Variables}

\begin{itemize}
    \item \textbf{pathname}: A string environment variable representing the
    file's full path or filename to be accessed by the Hardware-Hiding Module.
\end{itemize}

\subsubsection{Assumptions}

\begin{itemize}
\item The input parameters to open must be valid and correspond to existing
hardware components.
\item The write operation assumes that the hardware device is available and
writable.
\item The read operation assumes that the hardware device is readable.
\end{itemize}

\subsubsection{Access Routine Semantics}

\noindent open(pathname, flags, mode):

See \url{https://en.wikipedia.org/wiki/Open_(system_call)}.

\noindent write(fd, buf, count):

See \url{https://en.wikipedia.org/wiki/Write_(system_call)}.

\noindent read(fd, buf, count):

See \url{https://en.wikipedia.org/wiki/Read_(system_call)}.

\subsubsection{Local Functions}

None

\newpage

\section{MIS of General Digital Filter Module} \label{MIS_GDF}

This section outlines the formal specification of the General Digital Filter
Module, which provides linear filter functions for filtering in the form of
difference equations.

\subsection{Module}

General Digital Filter

\subsection{Uses}

\begin{itemize}
\item Error Handler Module (Section \ref{MIS_ERROR})
\item Logger Module (Section \ref{MIS_Logger})
\end{itemize}

\subsection{Syntax}

This section describes the syntax used in the General Digital Filter Module,
including the constants, programs, and routines that can be accessed.

\subsubsection{Exported Constants}

None

\subsubsection{Exported Access Programs}

\begin{center}
\begin{tabular}{p{3cm} p{4cm} p{4cm} p{2cm}}
\hline
\textbf{Name} & \textbf{In} & \textbf{Out} & \textbf{Exceptions} \\
\hline
lfilter & $\mathbb{R}^n$, $\mathbb{R}^m$, $\mathbb{R}^t$ & $\mathbb{R}^t$ & - \\
\hline
\end{tabular}
\end{center}

\subsection{Semantics}

This section provides the detailed functionality and behavior of the module’s
functions.

\subsubsection{State Variables}

None

\subsubsection{Environment Variables}

None

\subsubsection{Assumptions}

\begin{itemize}
\item The filter numerator coefficients $b = [b_0, b_1, \dots, b_{n-1}]$ and
denominator coefficients $a = [a_0, a_1, \dots, a_{m-1}]$ are valid real-valued
sequences.
\item The input sequence $x = [x_0, x_1, \dots, x_{t-1}]$ is a real-valued
discrete-time series.
\item The first coefficient of $a$ satisfies $a_0 \neq 0$ to ensure the
difference equation is well-defined.
\end{itemize}

\subsubsection{Access Routine Semantics}

\noindent lfilter(b, a, x):
\begin{itemize}
\item output: The filtered sequence $y$.
\item exception: None
\end{itemize}

\subsubsection{Local Functions}

None


\section{MIS of Specified IIR Filter Module} \label{MIS_SIIR}

This section outlines the formal specification of the Specified IIR Filter
Module, which provides parameter derivation functions for Butterworth and
Chebyshev filters.

\subsection{Module}

Specified IIR Filter

\subsection{Uses}

\begin{itemize}
\item General Digital Filter Module (Section \ref{MIS_GDF})
\item Error Handler Module (Section \ref{MIS_ERROR})
\item Logger Module (Section \ref{MIS_Logger})
\end{itemize}

\subsection{Syntax}

This section describes the syntax used in the Specified IIR Filter Module,
including the constants, programs, and routines that can be accessed.

\subsubsection{Exported Constants}

None

\subsubsection{Exported Access Programs}

\begin{center}
\begin{tabular}{p{2cm} p{4cm} p{4cm} p{5cm}}
\hline
\textbf{Name} & \textbf{In} & \textbf{Out} & \textbf{Exceptions} \\
\hline
butter & $\mathbb{N}$, $\mathbb{R}$, $\text{string}$ & $\mathbb{R}^{n}$,
$\mathbb{R}^{m}$ & ResultExceededExpectation \\
cheby1 & $\mathbb{N}$, $\mathbb{R}$, $\mathbb{R}$, $\text{string}$ &
$\mathbb{R}^{n}$, $\mathbb{R}^{m}$ & ResultExceededExpectation \\
cheby2 & $\mathbb{N}$, $\mathbb{R}$, $\mathbb{R}$, $\text{string}$ &
$\mathbb{R}^{n}$, $\mathbb{R}^{m}$ & ResultExceededExpectation \\
\hline
\end{tabular}
\end{center}

\subsection{Semantics}

This section provides the detailed functionality and behavior of the module’s
functions.

\subsubsection{State Variables}

None

\subsubsection{Environment Variables}

\begin{itemize}
\item \textbf{system\_epsilon}: The floating-point precision threshold of the
host system, used to determine acceptable numerical error bounds in filter
coefficient calculation.
\item \textbf{max\_order\_limit}: The maximum allowable filter order supported
by the system to prevent excessive computational cost or numerical instability.
\end{itemize}

\subsubsection{Assumptions}

\begin{itemize}
\item The system performing the calculations can handle floating-point precision
properly and maintains \texttt{system\_epsilon} within a reasonable range for
numerical stability.
\item The requested filter order does not exceed \texttt{max\_order\_limit}.
\end{itemize}

\subsubsection{Access Routine Semantics}

\noindent butter(N, Wn, btype):
\begin{itemize}
\item transition:
    \begin{itemize}
    \item Verify that $N \leq \texttt{max\_order\_limit}$; if not, log the event
    and raise \texttt{ResultExceededExpectationError}.
    \item Compute the Butterworth filter poles in the s-domain.
    \item Apply a bilinear transformation to map the s-domain poles to the
    z-domain.
    \item Convert the poles to numerator ($b$) and denominator ($a$) sequences.
    \item Ensure that numerical errors in $b$ and $a$ are within
    \texttt{system\_epsilon}; if not, log and raise
    \texttt{ResultExceededExpectationError}.
    \end{itemize}
\item output: The parameter sequences $b$ and $a$ of the filter.
\item exception: 
    \begin{itemize}
    \item If $N > \texttt{max\_order\_limit}$.
    \item If the resulting coefficients contain numerical errors exceeding
    \texttt{system\_epsilon}.
    \end{itemize}
\end{itemize}

\noindent cheby1(N, rp, Wn, btype):
\begin{itemize}
\item transition:
    \begin{itemize}
    \item Validate $N \leq \texttt{max\_order\_limit}$.
    \item Calculate Chebyshev Type I poles with ripple $rp$.
    \item Transform the poles from the s-domain to the z-domain via bilinear
    transformation.
    \item Derive $b$ and $a$ coefficient sequences.
    \item Check if computed coefficients meet precision threshold
    \texttt{system\_epsilon}; otherwise, raise exception.
    \end{itemize}
\item output: The parameter sequences $b$ and $a$ of the filter.
\item exception:
    \begin{itemize}
    \item If $N > \texttt{max\_order\_limit}$.
    \item If numerical errors in results exceed \texttt{system\_epsilon}.
    \end{itemize}
\end{itemize}

\noindent cheby2(N, rs, Wn, btype):
\begin{itemize}
\item transition:
    \begin{itemize}
    \item Validate $N \leq \texttt{max\_order\_limit}$.
    \item Calculate Chebyshev Type II poles ensuring stopband attenuation $rs$.
    \item Apply bilinear transformation for s-to-z domain mapping.
    \item Form coefficient sequences $b$ and $a$.
    \item Check numerical stability and precision against
    \texttt{system\_epsilon}.
    \end{itemize}
\item output: The parameter sequences $b$ and $a$ of the filter.
\item exception:
    \begin{itemize}
    \item If $N > \texttt{max\_order\_limit}$.
    \item If precision violations occur beyond \texttt{system\_epsilon}.
    \end{itemize}
\end{itemize}

\subsubsection{Local Functions}

None

\newpage

\section{MIS of Math Module} \label{MIS_Math}

This section outlines the formal specification of the Math Module, which
provides three mathematical functions: squaring, thresholding, and RMSE
calculation.

\subsection{Module}

Math

\subsection{Uses}

\begin{itemize}
\item Error Handler Module (Section \ref{MIS_ERROR})
\item Logger Module (Section \ref{MIS_Logger})
\end{itemize}

\subsection{Syntax}

This section describes the syntax used in the Math Module, including the
constants, programs, and routines that can be accessed.

\subsubsection{Exported Constants}

None

\subsubsection{Exported Access Programs}

\begin{center}
\begin{tabular}{p{2.5cm} p{4.5cm} p{4.5cm} p{2.5cm}}
\hline
\textbf{Name} & \textbf{In} & \textbf{Out} & \textbf{Exceptions} \\
\hline
calSquare & $\mathbb{R}^n$ & $\mathbb{R}^n$ & - \\
calThreshold & $\mathbb{R}^n$, $\mathbb{R}^n$ & $\mathbb{R}^n$ & InvalidInput \\
calRMSE & $\mathbb{R}^n$, $\mathbb{R}^n$ & $\mathbb{R}$ & InvalidInput \\
\hline
\end{tabular}
\end{center}

\subsection{Semantics}

This section provides the detailed functionality and behavior of the module’s
functions.

\subsubsection{State Variables}

None

\subsubsection{Environment Variables}

\begin{itemize}
\item \textbf{system\_epsilon}: The floating-point precision threshold of the
host system, used to determine acceptable numerical error in RMSE calculation.
\end{itemize}

\subsubsection{Assumptions}

\begin{itemize}
\item All input sequences are non-empty.
\item Floating-point operations on the host system conform to the system-defined
\texttt{system\_epsilon}.
\end{itemize}

\subsubsection{Access Routine Semantics}

\noindent calSquare(x):
\begin{itemize}
\item transition:
    \begin{itemize}
    \item For each element $x_i$ in sequence $x$, compute $x_i^2$.
    \item Store the result in the corresponding position in the output sequence.
    \end{itemize}
\item output: A sequence containing the square of each element from $x$.
\end{itemize}

\noindent calThreshold(x, thres):
\begin{itemize}
\item transition:
    \begin{itemize}
    \item Verify that the lengths of $x$ and \texttt{thres} are equal.
    \item For each pair of elements $(x_i, \text{thres}_i)$:
        \begin{itemize}
        \item If $x_i \geq \text{thres}_i$, retain $x_i$.
        \item Otherwise, set the result to 0.
        \end{itemize}
    \end{itemize}
\item output: A sequence of thresholded values.
\item exception:
    \begin{itemize}
    \item If length of $x$ and \texttt{thres} differ, log the event and raise
    \texttt{InvalidInput} exception.
    \end{itemize}
\end{itemize}

\noindent calRMSE(x, y):
\begin{itemize}
\item transition:
    \begin{itemize}
    \item Verify that the lengths of $x$ and $y$ are equal.
    \item Compute the mean of the squared differences:
        \[
        \text{MSE} = \frac{1}{n} \sum_{i=1}^{n} (x_i - y_i)^2
        \]
    \item Compute the square root of MSE to obtain RMSE.
    \item Check if the final RMSE value contains invalid floating-point values
    (e.g. NaN or infinity) or violates \texttt{system\_epsilon} tolerance; if
    so, log and raise \texttt{InvalidInput}.
    \end{itemize}
\item output: The calculated RMSE value.
\item exception:
    \begin{itemize}
    \item If length of $x$ and $y$ differ, or result precision is invalid, raise
    \texttt{InvalidInput} exception.
    \end{itemize}
\end{itemize}

\subsubsection{Local Functions}

None

\newpage

\section{MIS of Pan Tompkins Algorithm Module} \label{MIS_Alg}

This section outlines the formal specification of the Pan Tompkins Algorithm
Module, which provides the implementation of the Pan-Tompkins algorithm.

\subsection{Module}

Pan Tompkins Algorithm

\subsection{Uses}

\begin{itemize}
\item Specified IIR Filter Module (Section \ref{MIS_SIIR})
\item General Digital Filter Module (Section \ref{MIS_GDF})
\item Math Module (Section \ref{MIS_Math})
\item Error Handler Module (Section \ref{MIS_ERROR})
\item Logger Module (Section \ref{MIS_Logger})
\end{itemize}

\subsection{Syntax}

This section describes the syntax used in the Pan Tompkins Algorithm Module,
including the constants, programs, and routines that can be accessed.

\subsubsection{Exported Constants}

None

\subsubsection{Exported Access Programs}

\begin{center}
\begin{tabular}{p{2.5cm} p{4.5cm} p{4.5cm} p{2.5cm}}
\hline
\textbf{Name} & \textbf{In} & \textbf{Out} & \textbf{Exceptions} \\
\hline
findPeak & $\mathbb{R}^n$, $\mathbb{N}$ & $\mathbb{Z}^m$ & InvalidInput \\
\hline
\end{tabular}
\end{center}

\subsection{Semantics}

This section provides the detailed functionality and behavior of the module’s
functions.

\subsubsection{State Variables}

None

\subsubsection{Environment Variables}

\begin{itemize}
\item \textbf{lowpass\_cutoff}: Low-pass filter cutoff frequency (Hz).
\item \textbf{highpass\_cutoff}: High-pass filter cutoff frequency (Hz).
\item \textbf{integration\_window}: Window size for moving average integration
(in samples).
\item \textbf{peak\_threshold\_factor}: Multiplier for adaptive peak detection
threshold.
\end{itemize}

\subsubsection{Assumptions}

\begin{itemize}
\item The input sequence \texttt{rawSignal} is a non-empty array of real
numbers.
\item The input sequence \texttt{rawSignal} is suitable for digital filtering.
\item Sampling frequency \texttt{fs} is a positive integer.
\end{itemize}

\subsubsection{Access Routine Semantics}

\noindent findPeak(rawSignal, fs):
\begin{itemize}
\item transition:
    \begin{itemize}
    \item Verify that the length of \texttt{rawSignal} is greater than the
    required minimum based on filter orders and integration window size.
    \item Use \texttt{Specified IIR Filter Module} to design Butterworth
    bandpass filters with \texttt{lowpass\_cutoff} and
    \texttt{highpass\_cutoff}, normalized by \texttt{fs}.
    \item Apply bandpass filtering to \texttt{rawSignal} using \texttt{General
    Digital Filter Module}.
    \item Differentiate the filtered signal.
    \item Square each sample in the differentiated signal using \texttt{Math
    Module: calSquare}.
    \item Apply moving average integration using a window of size
    \texttt{integration\_window}.
    \item Detect peaks by:
        \begin{itemize}
        \item Computing an adaptive threshold as a fraction
        (\texttt{peak\_threshold\_factor}) of the maximum value in the
        integrated signal.
        \item Identifying local maxima exceeding this threshold.
        \end{itemize}
    \item Log the number of detected peaks and their locations.
    \end{itemize}
\item output: An integer sequence representing the indices of R-wave peaks.
\item exception:
    \begin{itemize}
    \item If \texttt{rawSignal} is shorter than the required minimum length, log
    the event and raise \texttt{InvalidInput}.
    \end{itemize}
\end{itemize}

\subsubsection{Local Functions}

None

\newpage

\section{MIS of Annotated Data RMSE Module} \label{MIS_RMSE}

This section outlines the formal specification of the Annotated Data RMSE
Module, which provides functionality to compare algorithm results with annotated
data to determine the magnitude of algorithmic errors.

\subsection{Module}

Annotated Data RMSE

\subsection{Uses}

\begin{itemize}
\item Math Module (Section \ref{MIS_Math})
\item Error Handler Module (Section \ref{MIS_ERROR})
\item Logger Module (Section \ref{MIS_Logger})
\end{itemize}

\subsection{Syntax}

This section describes the syntax used in the Annotated Data RMSE Module,
including the constants, programs, and routines that can be accessed.

\subsubsection{Exported Constants}

None

\subsubsection{Exported Access Programs}

\begin{center}
\begin{tabular}{p{2.5cm} p{4.5cm} p{4.5cm} p{3.5cm}}
\hline
\textbf{Name} & \textbf{In} & \textbf{Out} & \textbf{Exceptions} \\
\hline
signalRMSE & $\mathbb{Z}^n$, $\mathbb{Z}^m$ & $\mathbb{R}$ & NoMatchingPoint \\
\hline
\end{tabular}
\end{center}

\subsection{Semantics}

This section provides the detailed functionality and behavior of the module’s
functions.

\subsubsection{State Variables}

None

\subsubsection{Environment Variables}

\begin{itemize}
\item \textbf{matching\_window}: Maximum allowable difference (in samples)
between a raw index and an annotation index for them to be considered a match.
\item \textbf{min\_matching\_count}: Minimum number of matched point pairs
required to compute RMSE.
\end{itemize}

\subsubsection{Assumptions}

\begin{itemize}
\item The implementation of the Pan-Tompkins algorithm is correct.
\item Both \texttt{rawIndices} and \texttt{annIndices} are sorted in ascending
order.
\end{itemize}

\subsubsection{Access Routine Semantics}

\noindent signalRMSE(rawIndices, annIndices):
\begin{itemize}
\item transition:
    \begin{itemize}
    \item Match each annotated index in \texttt{annIndices} with the nearest raw
    index in \texttt{rawIndices}, provided the absolute difference is less than
    or equal to \texttt{matching\_window}.
    \item Form pairs of matched points.
    \item If the number of matched pairs is less than
    \texttt{min\_matching\_count}, log this event and raise
    \texttt{NoMatchingPoint} exception.
    \item Compute the differences between matched pairs.
    \item Use \texttt{Math Module: calRMSE} to compute the root mean square
    error (RMSE) from these differences.
    \item Log the final RMSE value.
    \end{itemize}
\item output: A real number representing the RMSE between matched points.
\item exception:
    \begin{itemize}
    \item If no matching pairs are found or the number of matches is below
    \texttt{min\_matching\_count}, raise \texttt{NoMatchingPoint}.
    \end{itemize}
\end{itemize}

\subsubsection{Local Functions}

None

\newpage

\section{MIS of Rwave Detect Module} \label{MIS_Rwave}

This section outlines the formal specification of the Rwave Detect Module, which
provides initialization and invocation functionalities for other modules,
serving as the entry point for the entire program.

\subsection{Module}

Rwave Detect

\subsection{Uses}

\begin{itemize}
\item Input Output Processing Module (Section \ref{MIS_IO})
\item Pan Tompkins Algorithm Module (Section \ref{MIS_Alg})
\item Annotated Data RMSE Module (Section \ref{MIS_RMSE})
\item Error Handler Module (Section \ref{MIS_ERROR})
\item Logger Module (Section \ref{MIS_Logger})
\end{itemize}

\subsection{Syntax}

This section describes the syntax used in the Rwave Detect Module, including the
constants, programs, and routines that can be accessed.

\subsubsection{Exported Constants}

None

\subsubsection{Exported Access Programs}

\begin{center}
\begin{tabular}{p{2.5cm} p{4.5cm} p{4.5cm} p{3.5cm}}
\hline
\textbf{Name} & \textbf{In} & \textbf{Out} & \textbf{Exceptions} \\
\hline
mainEntry & - & - & - \\
\hline
\end{tabular}
\end{center}

\subsection{Semantics}

This section provides the detailed functionality and behavior of the module’s
functions.

\subsubsection{State Variables}

\begin{itemize}
\item \textbf{runningState}: Records the current stage of the program's
execution, such as \textbf{Initializing}, \textbf{Running}, or
\textbf{ErrorTerminated}.
\end{itemize}

\subsubsection{Environment Variables}

None

\subsubsection{Assumptions}

\begin{itemize}
\item The program will first initialize and enter this module during execution.
\item All dependent modules are correctly implemented and can be initialized
without failure.
\end{itemize}

\subsubsection{Access Routine Semantics}

\noindent mainEntry:
\begin{itemize}
\item transition:
    \begin{enumerate}
    \item Set \texttt{runningState} to \texttt{Initializing}.
    \item Log the program start event.
    \item Initialize all dependent modules:
        \begin{itemize}
        \item Input Output Processing Module
        \item Pan Tompkins Algorithm Module
        \item Annotated Data RMSE Module
        \item Error Handler Module
        \item Logger Module
        \end{itemize}
    \item Load raw ECG data using the Input Output Processing Module.
    \item Apply the Pan-Tompkins Algorithm Module to detect R-wave peak indices.
    \item Load annotated (reference) R-wave peak indices.
    \item Call the Annotated Data RMSE Module to calculate the RMSE between
    detected and annotated indices.
    \item Output the RMSE result using the Input Output Processing Module.
    \item Set \texttt{runningState} to \texttt{Running}.
    \item Log successful completion.
    \end{enumerate}
\item exception:
    \begin{itemize}
    \item If any unexpected error or exception occurs during module
    initialization or execution, log the error via the Logger Module and set
    \texttt{runningState} to \texttt{ErrorTerminated}.
    \item Use the Error Handler Module to gracefully handle or propagate
    exceptions as necessary.
    \end{itemize}
\end{itemize}

\subsubsection{Local Functions}

None

\newpage

\section{MIS of Input Output Processing Module} \label{MIS_IO}

This section outlines the formal specification of the Input Output Processing
Module, which provides basic data input and output functionality by calling the
Hardware-Hiding Module.

\subsection{Module}

Input Output Processing

\subsection{Uses}

\begin{itemize}
\item Hardware-Hiding Module (Section \ref{MIS_HH})
\item Error Handler Module (Section \ref{MIS_ERROR})
\item Logger Module (Section \ref{MIS_Logger})
\end{itemize}

\subsection{Syntax}

This section describes the syntax used in the Input Output Processing Module,
including the constants, programs, and routines that can be accessed.

\subsubsection{Exported Constants}

None

\subsubsection{Exported Access Programs}

\begin{center}
\begin{tabular}{p{3cm} p{4cm} p{4cm} p{2cm}}
\hline
\textbf{Name} & \textbf{In} & \textbf{Out} & \textbf{Exceptions} \\
\hline
readFromFile & string & $\mathbb{R}^n$ & FileError \\
writeToFile & string, $\mathbb{Z}^m$ & - & FileError \\
\hline
\end{tabular}
\end{center}

\subsection{Semantics}

This section provides the detailed functionality and behavior of the module’s
functions.

\subsubsection{State Variables}

None

\subsubsection{Environment Variables}

None

\subsubsection{Assumptions}

\begin{itemize}
\item The operating system supports correct file read and write operations.
\item The file paths provided are valid and accessible to the program.
\end{itemize}

\subsubsection{Access Routine Semantics}

\noindent readFromFile(filename):
\begin{itemize}
\item transition: 
    \begin{itemize}
    \item Calls the Hardware-Hiding Module to open and read the file located at
    the specified \texttt{filename}.
    \item Parses the file content into an array of real numbers representing the
    ECG signal.
    \end{itemize}
\item output: A sequence of real numbers $\mathbb{R}^n$ representing the ECG
signal after format conversion.
\item exception: 
    \begin{itemize}
    \item If the file cannot be opened.
    \item If the file format is invalid or unreadable.
    \end{itemize}
    Then throw a \texttt{FileError} exception.
\end{itemize}

\noindent writeToFile(filename, index):
\begin{itemize}
\item transition:
    \begin{itemize}
    \item Calls the Hardware-Hiding Module to open or create the file specified
    by \texttt{filename}.
    \item Writes the integer sequence \texttt{index} (R-wave peak indices) into
    the file, one value per line or in a defined format.
    \end{itemize}
\item output: None
\item exception:
    \begin{itemize}
    \item If the file cannot be opened for writing.
    \item If a write operation fails.
    \end{itemize}
    Then throw a \texttt{FileError} exception.
\end{itemize}

\subsubsection{Local Functions}

None

\newpage

\section{MIS of Error Handler Module} \label{MIS_ERROR}

This section outlines the formal specification of the Error Handler Module,
which provides handling functionality for exceptions that occur during the
execution of the program.

\subsection{Module}

Error Handler

\subsection{Uses}

\begin{itemize}
\item Hardware-Hiding Module (Section \ref{MIS_HH})
\end{itemize}

\subsection{Syntax}

This section describes the syntax used in the Error Handler Module, including
the constants, programs, and routines that can be accessed.

\subsubsection{Exported Constants}

None

\subsubsection{Exported Access Programs}

\begin{center}
\begin{tabular}{p{3cm} p{4cm} p{4cm} p{2cm}}
\hline
\textbf{Name} & \textbf{In} & \textbf{Out} & \textbf{Exceptions} \\
\hline
throwException & string & - & - \\
\hline
\end{tabular}
\end{center}

\subsection{Semantics}

This section provides the detailed functionality and behavior of the module’s
functions.

\subsubsection{State Variables}

None

\subsubsection{Environment Variables}

None

\subsubsection{Assumptions}

\begin{itemize}
\item The exception information can be correctly logged and saved via the
Hardware-Hiding Module.
\item The program can safely terminate or recover following an exception.
\end{itemize}

\subsubsection{Access Routine Semantics}

\noindent throwException(ex):
\begin{itemize}
\item transition:
    \begin{itemize}
    \item Immediately log the exception message \texttt{ex} using the
    Hardware-Hiding Module.
    \item Optionally notify the user via the standard output or log file.
    \item Terminate the program’s execution or transfer control to a safe
    shutdown routine.
    \end{itemize}
\item output: None
\item exception: None
\end{itemize}

\subsubsection{Local Functions}

None


\section{MIS of Logger Module} \label{MIS_Logger}

This section outlines the formal specification of the Logger Module, which
provides functionality for logging output.

\subsection{Module}

Logger

\subsection{Uses}

\begin{itemize}
\item Hardware-Hiding Module (Section \ref{MIS_HH})
\end{itemize}

\subsection{Syntax}

This section describes the syntax used in the Logger Module, including the
constants, programs, and routines that can be accessed.

\subsubsection{Exported Constants}

None

\subsubsection{Exported Access Programs}

\begin{center}
\begin{tabular}{p{3cm} p{4cm} p{4cm} p{2cm}}
\hline
\textbf{Name} & \textbf{In} & \textbf{Out} & \textbf{Exceptions} \\
\hline
log & string, $\mathbb{N}$ & - & - \\
setLogLevel & $\mathbb{N}$ & - & - \\
setLogPath & string & - & FileError \\
\hline
\end{tabular}
\end{center}

\subsection{Semantics}

This section provides the detailed functionality and behavior of the module’s
functions.

\subsubsection{State Variables}

\begin{itemize}
\item runningTime: The runtime of the program since initialization.
\item logLevel: The current log level threshold. Messages below this level will
not be logged.
\item logPath: The path to the log file.
\end{itemize}

\subsubsection{Environment Variables}

\begin{itemize}
\item The default log level in the configuration file.
\end{itemize}

\subsubsection{Assumptions}

\begin{itemize}
\item Logs can be correctly output and stored via the Hardware-Hiding Module.
\item The file system is accessible for writing log files.
\end{itemize}

\subsubsection{Access Routine Semantics}

\noindent log(msg, level):
\begin{itemize}
\item transition:
    \begin{itemize}
    \item If \texttt{level} $\geq$ \texttt{logLevel}, format the message with a
    timestamp and program \texttt{runningTime}, and append it to the log file at
    \texttt{logPath}.
    \item If \texttt{level} $<$ \texttt{logLevel}, no action is performed.
    \end{itemize}
\item output: None
\item exception: None
\end{itemize}

\noindent setLogLevel(level):
\begin{itemize}
\item transition: Set \texttt{logLevel} to the specified \texttt{level}.
Suppress all log messages with a level lower than this value.
\item output: None
\item exception: None
\end{itemize}

\noindent setLogPath(filename):
\begin{itemize}
\item transition: Set \texttt{logPath} to \texttt{filename}. All subsequent log
messages will be written to this file.
\item exception: If the file cannot be opened for writing, throw a
\texttt{FileError}.
\end{itemize}

\subsubsection{Log Levels Definition}

The following numeric levels are used to categorize log messages:

\begin{itemize}
\item \textbf{0}: ERROR — Critical issues causing program termination.
\item \textbf{1}: WARNING — Important events that do not halt the program but
require attention.
\item \textbf{2}: INFO — General information about program execution.
\item \textbf{3}: DEBUG — Detailed debugging messages for development use.
\end{itemize}

\subsubsection{Local Functions}

None

\newpage

\bibliographystyle {plainnat}
\bibliography {../../../refs/References}

% \newpage

% \section{Appendix} \label{Appendix}

% \wss{Extra information if required}

\end{document}