\documentclass[12pt, titlepage]{article}

\usepackage{amsmath, mathtools}

\usepackage[round]{natbib}
\usepackage{amsfonts}
\usepackage{amssymb}
\usepackage{graphicx}
\usepackage{colortbl}
\usepackage{xr}
\usepackage{hyperref}
\usepackage{longtable}
\usepackage{xfrac}
\usepackage{tabularx}
\usepackage{float}
\usepackage{siunitx}
\usepackage{booktabs}
\usepackage{multirow}
\usepackage[section]{placeins}
\usepackage{caption}
\usepackage{fullpage}

\hypersetup{ bookmarks=true,     % show bookmarks bar?
colorlinks=true,       % false: boxed links; true: colored links
linkcolor=red,          % color of internal links (change box color with
                        % linkbordercolor)
citecolor=blue,      % color of links to bibliography
filecolor=magenta,  % color of file links
urlcolor=cyan          % color of external links
}

\usepackage{array}

\externaldocument{../../SRS/SRS}

%% Comments

\usepackage{color}

\newif\ifcomments\commentstrue %displays comments
%\newif\ifcomments\commentsfalse %so that comments do not display

\ifcomments
\newcommand{\authornote}[3]{\textcolor{#1}{[#3 ---#2]}}
\newcommand{\todo}[1]{\textcolor{red}{[TODO: #1]}}
\else
\newcommand{\authornote}[3]{}
\newcommand{\todo}[1]{}
\fi

\newcommand{\wss}[1]{\authornote{blue}{SS}{#1}} 
\newcommand{\plt}[1]{\authornote{magenta}{TPLT}{#1}} %For explanation of the template
\newcommand{\an}[1]{\authornote{cyan}{Author}{#1}}

%% Common Parts

\newcommand{\progname}{ProgName} % PUT YOUR PROGRAM NAME HERE
\newcommand{\authname}{Team \#, Team Name
\\ Student 1 name
\\ Student 2 name
\\ Student 3 name
\\ Student 4 name} % AUTHOR NAMES                  

\usepackage{hyperref}
    \hypersetup{colorlinks=true, linkcolor=blue, citecolor=blue, filecolor=blue,
                urlcolor=blue, unicode=false}
    \urlstyle{same}
                                


\begin{document}

\title{Module Interface Specification for \progname{}}

\author{\authname}

\date{\today}

\maketitle

\pagenumbering{roman}

\section{Revision History}

\begin{tabularx}{\textwidth}{p{3cm}p{2cm}X} \toprule {\bf Date} & {\bf Version}
& {\bf Notes}\\
\midrule
Mar 17, 2025 & 1.0 & Creation\\
\bottomrule
\end{tabularx}

~\newpage

\section{Symbols, Abbreviations and Acronyms}

See SRS Documentation at
\url{https://github.com/Lychee-acaca/CAS741/blob/main/docs/SRS/SRS.pdf}.

\wss{Also add any additional symbols, abbreviations or acronyms}

\newpage

\tableofcontents

\newpage

\pagenumbering{arabic}

\section{Introduction}

The following document details the Module Interface Specifications for
\progname.

This project focuses on reimplementing the Pan-Tompkins algorithm for R-wave
detection in ECG signal processing.

Complementary documents include the System Requirement Specifications and Module
Guide.  The full documentation and implementation can be found at
\url{https://github.com/Lychee-acaca/CAS741}.

\section{Notation}

\wss{You should describe your notation.  You can use what is below as a starting
  point.}

The structure of the MIS for modules comes from \citet{HoffmanAndStrooper1995},
with the addition that template modules have been adapted from
\cite{GhezziEtAl2003}.  The mathematical notation comes from Chapter 3 of
\citet{HoffmanAndStrooper1995}.  For instance, the symbol := is used for a
multiple assignment statement and conditional rules follow the form $(c_1
\Rightarrow r_1 | c_2 \Rightarrow r_2 | ... | c_n \Rightarrow r_n )$.

The following table summarizes the primitive data types used by \progname. 

\begin{center}
\renewcommand{\arraystretch}{1.2}
\noindent 
\begin{tabular}{l l p{7.5cm}} 
\toprule 
\textbf{Data Type} & \textbf{Notation} & \textbf{Description}\\ 
\midrule
character & $\text{char}$ & a single symbol or digit\\
integer & $\mathbb{Z}$ & a number without a fractional component in (-$\infty$,
$\infty$) \\
natural number & $\mathbb{N}$ & a number without a fractional component in [1,
$\infty$) \\
real & $\mathbb{R}$ & any number in (-$\infty$, $\infty$)\\
string & $\text{string}$ & -\\
\bottomrule
\end{tabular} 
\end{center}

\noindent
The specification of \progname \ uses some derived data types: sequences,
strings, and tuples. Sequences are lists filled with elements of the same data
type. Strings are sequences of characters. Tuples contain a list of values,
potentially of different types. In addition, \progname \ uses functions, which
are defined by the data types of their inputs and outputs. Local functions are
described by giving their type signature followed by their specification.

\section{Module Decomposition}

The following table is taken directly from the Module Guide document for this
project.

\begin{table}[h!]
\centering
\begin{tabular}{p{0.3\textwidth} p{0.6\textwidth}}
\toprule
\textbf{Level 1} & \textbf{Level 2}\\
\midrule

{Hardware-Hiding Module} & ~ \\
\midrule

\multirow{5}{0.3\textwidth}{Behaviour-Hiding Module}
& General Digital Filter Module \\
& Specified IIR Filter Module \\
& Math Module \\
& Pan Tompkins Algorithm Module \\
& Annotated Data RMSE Module \\
\midrule

\multirow{4}{0.3\textwidth}{Software Decision Module}
& Rwave Detect Module \\
& Input Output Processing Module \\
& Error Handler Module \\
& Logger Module \\
\bottomrule

\end{tabular}
\caption{Module Hierarchy}
\label{TblMH}
\end{table}

\newpage
~\newpage

\section{MIS of Hardware-Hiding Module} \label{MIS_HH}

This section outlines the formal specification of the Hardware-Hiding Module,
which provides an abstraction layer for interacting with hardware devices.

\subsection{Module}

Hardware-Hiding

\subsection{Uses}

None

\subsection{Syntax}

This section describes the syntax used in the Hardware-Hiding Module, including
the constants, programs, and routines that can be accessed.

\subsubsection{Exported Constants}

None

\subsubsection{Exported Access Programs}

\begin{center}
\begin{tabular}{p{2cm} p{4cm} p{4cm} p{2cm}}
\hline
\textbf{Name} & \textbf{In} & \textbf{Out} & \textbf{Exceptions} \\
\hline
open & $\text{string}$, $\mathbb{N}$, $\mathbb{N}$ & $\mathbb{Z}$ & - \\
write & $\mathbb{Z}$, $\mathbb{N}^n$, $\mathbb{N}$ & $\mathbb{N}$ & - \\
read & $\mathbb{Z}$, $\mathbb{N}^n$, $\mathbb{N}$ & $\mathbb{N}$ & - \\
\hline
\end{tabular}
\end{center}

\subsection{Semantics}

This section provides the detailed functionality and behavior of the module’s
functions.

\subsubsection{State Variables}

This module does not maintain state variables.

\subsubsection{Environment Variables}

This module interacts with hardware devices, such as file systems or external
I/O devices.

\subsubsection{Assumptions}

\begin{itemize}
\item The input parameters to open must be valid and correspond to existing
hardware components.
\item The write operation assumes that the hardware device is available and
writable.
\item The read operation assumes that the hardware device is readable.
\end{itemize}

\subsubsection{Access Routine Semantics}

\noindent open(pathname, flags, mode):

See \url{https://en.wikipedia.org/wiki/Open_(system_call)}.

\noindent write(fd, buf, count):

See \url{https://en.wikipedia.org/wiki/Write_(system_call)}.

\noindent read(fd, buf, count):

See \url{https://en.wikipedia.org/wiki/Read_(system_call)}.

\subsubsection{Local Functions}

None

\newpage

\section{MIS of General Digital Filter Module} \label{MIS_GDF}

\subsection{Module}

General Digital Filter

\subsection{Uses}

\begin{itemize}
\item Error Handler Module
\item Logger Module
\end{itemize}

\subsection{Syntax}

This section describes the syntax used in the General Digital Filter Module,
including the constants, programs, and routines that can be accessed.

\subsubsection{Exported Constants}

None

\subsubsection{Exported Access Programs}

\begin{center}
\begin{tabular}{p{2cm} p{4cm} p{4cm} p{2cm}}
\hline
\textbf{Name} & \textbf{In} & \textbf{Out} & \textbf{Exceptions} \\
\hline
lfilter & $\mathbb{R}^n$, $\mathbb{R}^m$, $\mathbb{R}^t$ & $\mathbb{R}^t$ & - \\
\hline
\end{tabular}
\end{center}

\subsection{Semantics}

This section provides the detailed functionality and behavior of the module’s
functions.

\subsubsection{State Variables}

None

\subsubsection{Environment Variables}

None

\subsubsection{Assumptions}

\begin{itemize}
\item The filter coefficients $b$ and $a$ are valid real-valued sequences.
\item The input sequence $x$ is a real-valued time series of appropriate length.
\item The first coefficient of $a$ is nonzero to ensure system stability.
\end{itemize}

\subsubsection{Access Routine Semantics}

\noindent lfilter(b, a, x):
\begin{itemize}
\item transition: Calculate the filtered result of the sequence $x$ based on the
given filter parameters $b$ and $a$.
\item output: The result of the sequence $x$ after being filtered.
\end{itemize}

\subsubsection{Local Functions}

None

\newpage

\section{MIS of Specified IIR Filter Module} \label{MIS_SIIR}

\subsection{Module}

Specified IIR Filter

\subsection{Uses}

\begin{itemize}
\item General Digital Filter Module
\item Error Handler Module
\item Logger Module
\end{itemize}

\subsection{Syntax}

This section describes the syntax used in the Specified IIR Filter Module,
including the constants, programs, and routines that can be accessed.

\subsubsection{Exported Constants}

None

\subsubsection{Exported Access Programs}

\begin{center}
\begin{tabular}{p{2cm} p{4cm} p{4cm} p{2cm}}
\hline
\textbf{Name} & \textbf{In} & \textbf{Out} & \textbf{Exceptions} \\
\hline
butter & $\mathbb{N}$, $\mathbb{R}$, $\text{string}$ & $\mathbb{R}^{n}$,
$\mathbb{R}^{m}$ & - \\
cheby1 & $\mathbb{N}$, $\mathbb{R}$, $\mathbb{R}$, $\text{string}$ &
$\mathbb{R}^{n}$, $\mathbb{R}^{m}$ \\
cheby2 & $\mathbb{N}$, $\mathbb{R}$, $\mathbb{R}$, $\text{string}$ &
$\mathbb{R}^{n}$, $\mathbb{R}^{m}$ \\
\hline
\end{tabular}
\end{center}

\subsection{Semantics}

This section provides the detailed functionality and behavior of the module’s
functions.

\subsubsection{State Variables}

None

\subsubsection{Environment Variables}

None

\subsubsection{Assumptions}

\begin{itemize}
\item The system performing the calculations can handle floating-point precision
properly.
\end{itemize}

\subsubsection{Access Routine Semantics}

\noindent butter(N, Wn, btype):
\begin{itemize}
\item transition: Given the order $N$, cutoff angular frequency $W_n$, and type
of the Butterworth filter (high/low), calculate the parameters.
\item output: The parameter sequences $b$ and $a$ of the filter.
\item exception: 
\end{itemize}

\noindent cheby1(N, rp, Wn, btype):
\begin{itemize}
\item transition: Given the order $N$, the maximum ripple $rp$, the cutoff
angular frequency $W_n$, and type of the Chebyshev I filter (high/low),
calculate the parameters.
\item output: The parameter sequences $b$ and $a$ of the filter.
\item exception: 
\end{itemize}

\noindent cheby2(N, rs, Wn, btype):
\begin{itemize}
\item transition: Given the order $N$, the minimum attenuation $rs$, the cutoff
angular frequency $W_n$, and type of the Chebyshev II filter (high/low),
calculate the parameters.
\item output: The parameter sequences $b$ and $a$ of the filter.
\item exception: 
\end{itemize}

\subsubsection{Local Functions}

None

\newpage

\section{MIS of Math Module} \label{MIS_Math}

\subsection{Module}

Math

\subsection{Uses}

\begin{itemize}
\item Error Handler Module
\item Logger Module
\end{itemize}

\subsection{Syntax}

This section describes the syntax used in the Math Module, including the
constants, programs, and routines that can be accessed.

\subsubsection{Exported Constants}

None

\subsubsection{Exported Access Programs}

\begin{center}
\begin{tabular}{p{2cm} p{4cm} p{4cm} p{2cm}}
\hline
\textbf{Name} & \textbf{In} & \textbf{Out} & \textbf{Exceptions} \\
\hline
calSquare & $\mathbb{R}^n$ & $\mathbb{R}^n$ & - \\
calThreshold & $\mathbb{R}^n$, $\mathbb{R}^n$ & $\mathbb{R}^n$ & - \\
calRMSE & $\mathbb{R}^n$, $\mathbb{R}^n$ & $\mathbb{R}$ & - \\
\hline
\end{tabular}
\end{center}

\subsection{Semantics}

This section provides the detailed functionality and behavior of the module’s
functions.

\subsubsection{State Variables}

None

\subsubsection{Environment Variables}

None

\subsubsection{Assumptions}

\begin{itemize}
\item All input sequences are non-empty.
\item In calThreshold, the input sequences $x$ and $\text{thres}$ have the same
length.
\item In calRMSE, the sequences $x$ and $y$ are of equal length and contain
numerical values.
\end{itemize}

\subsubsection{Access Routine Semantics}

\noindent calSquare(x):
\begin{itemize}
\item transition: Input a sequence $x$, compute a new sequence where each
element is the square of the corresponding element in $x$.
\item output: The sequence after squaring.
\item exception: 
\end{itemize}

\noindent calThreshold(x, thres):
\begin{itemize}
\item transition: Input a sequence $x$ and a threshold sequence $\text{thres}$,
then compute the result of applying this threshold to $x$.
\item output: The result of applying the threshold to the sequence $x$.
\item exception: 
\end{itemize}

\noindent calRMSE(x, y):
\begin{itemize}
\item transition: Input two sequences $x$ and $y$ of the same length, then
compute the RMSE.
\item output: The RMSE of sequence $x$ and sequence $y$.
\item exception: 
\end{itemize}

\subsubsection{Local Functions}

None

\newpage

\section{MIS of Pan Tompkins Algorithm Module} \label{MIS_Alg}

\subsection{Module}

Pan Tompkins Algorithm

\subsection{Uses}

\begin{itemize}
\item Specified IIR Filter Module
\item General Digital Filter Module
\item Math Module
\item Error Handler Module
\item Logger Module
\end{itemize}

\subsection{Syntax}

This section describes the syntax used in the Pan Tompkins Algorithm Module,
including the constants, programs, and routines that can be accessed.

\subsubsection{Exported Constants}

None

\subsubsection{Exported Access Programs}

\begin{center}
\begin{tabular}{p{2cm} p{4cm} p{4cm} p{2cm}}
\hline
\textbf{Name} & \textbf{In} & \textbf{Out} & \textbf{Exceptions} \\
\hline
findPeak & $\mathbb{R}^n$, $\mathbb{N}$ & $\mathbb{Z}^m$ & - \\
\hline
\end{tabular}
\end{center}

\subsection{Semantics}

This section provides the detailed functionality and behavior of the module’s
functions.

\subsubsection{State Variables}

None

\subsubsection{Environment Variables}

None

\subsubsection{Assumptions}

\begin{itemize}
\item The input sequence rawSignal is a non-empty array of real numbers.
\item The input sequence rawSignal is filterable.
\end{itemize}

\subsubsection{Access Routine Semantics}

\noindent findPeak(rawSignal, fs):
\begin{itemize}
\item transition: Input the raw signal $\text{rawSignal}$ and the sampling
frequency $\text{fs}$, and use the Pan-Tompkins algorithm to compute the R-wave
peak indices.
\item output: An integer sequence representing the R-wave peak indices.
\item exception: 
\end{itemize}

\subsubsection{Local Functions}

None

\newpage

\section{MIS of Annotated Data RMSE Module} \label{MIS_RMSE}

\subsection{Module}

Annotated Data RMSE

\subsection{Uses}

\begin{itemize}
\item Math Module
\item Error Handler Module
\item Logger Module
\end{itemize}

\subsection{Syntax}

This section describes the syntax used in the Annotated Data RMSE Module,
including the constants, programs, and routines that can be accessed.

\subsubsection{Exported Constants}

None

\subsubsection{Exported Access Programs}

\begin{center}
\begin{tabular}{p{2cm} p{4cm} p{4cm} p{2cm}}
\hline
\textbf{Name} & \textbf{In} & \textbf{Out} & \textbf{Exceptions} \\
\hline
signalRMSE & $\mathbb{Z}^n$, $\mathbb{Z}^m$ & $\mathbb{R}$ & - \\
\hline
\end{tabular}
\end{center}

\subsection{Semantics}

This section provides the detailed functionality and behavior of the module’s
functions.

\subsubsection{State Variables}

None

\subsubsection{Environment Variables}

None

\subsubsection{Assumptions}

\begin{itemize}
\item a1
\end{itemize}

\subsubsection{Access Routine Semantics}

\noindent signalRMSE(rawIndices, annIndices):
\begin{itemize}
\item transition: Input two index sequences rawIndices and annIndices (possibly
of unequal lengths), match the nearby peak points, and compute the RMSE for the
matched points.
\item output: A real number representing the RMSE.
\item exception: 
\end{itemize}

\subsubsection{Local Functions}

None

\newpage

\section{MIS of Rwave Detect Module} \label{MIS_Rwave}

\subsection{Module}

Rwave Detect

\subsection{Uses}

\begin{itemize}
\item Input Output Processing Module
\item Pan Tompkins Algorithm Module
\item Annotated Data RMSE Module
\item Error Handler Module
\item Logger Module
\end{itemize}

\subsection{Syntax}

This section describes the syntax used in the Rwave Detect Module, including the
constants, programs, and routines that can be accessed.

\subsubsection{Exported Constants}

None

\subsubsection{Exported Access Programs}

\begin{center}
\begin{tabular}{p{2cm} p{4cm} p{4cm} p{2cm}}
\hline
\textbf{Name} & \textbf{In} & \textbf{Out} & \textbf{Exceptions} \\
\hline
mainEntry & - & - & - \\
\hline
\end{tabular}
\end{center}

\subsection{Semantics}

This section provides the detailed functionality and behavior of the module’s
functions.

\subsubsection{State Variables}

None

\subsubsection{Environment Variables}

None

\subsubsection{Assumptions}

\begin{itemize}
\item The program will first initialize and enter this module during execution.
\end{itemize}

\subsubsection{Access Routine Semantics}

\noindent mainEntry:
\begin{itemize}
\item transition: As the entry point of the program, it will initialize other
modules and make the corresponding function calls, coordinating the various
modules.
\end{itemize}

\subsubsection{Local Functions}

None

\newpage

\section{MIS of Input Output Processing Module} \label{MIS_IO}

\subsection{Module}

Input Output Processing

\subsection{Uses}

\begin{itemize}
\item Hardware-Hiding Module
\item Error Handler Module
\item Logger Module
\end{itemize}

\subsection{Syntax}

This section describes the syntax used in the Input Output Processing Module,
including the constants, programs, and routines that can be accessed.

\subsubsection{Exported Constants}

None

\subsubsection{Exported Access Programs}

\begin{center}
\begin{tabular}{p{2cm} p{4cm} p{4cm} p{2cm}}
\hline
\textbf{Name} & \textbf{In} & \textbf{Out} & \textbf{Exceptions} \\
\hline
- & - & - & - \\
\hline
\end{tabular}
\end{center}

\subsection{Semantics}

This section provides the detailed functionality and behavior of the module’s
functions.

\subsubsection{State Variables}

None

\subsubsection{Environment Variables}

None

\subsubsection{Assumptions}

\begin{itemize}
\item a1
\end{itemize}

\subsubsection{Access Routine Semantics}

\noindent a:
\begin{itemize}
\item transition: 
\item output: 
\item exception: 
\end{itemize}

\subsubsection{Local Functions}

None

\newpage

\section{MIS of Error Handler Module} \label{MIS_ERROR}

\subsection{Module}

Error Handler

\subsection{Uses}

\begin{itemize}
\item Hardware-Hiding Module
\end{itemize}

\subsection{Syntax}

This section describes the syntax used in the Error Handler Module, including
the constants, programs, and routines that can be accessed.

\subsubsection{Exported Constants}

None

\subsubsection{Exported Access Programs}

\begin{center}
\begin{tabular}{p{2cm} p{4cm} p{4cm} p{2cm}}
\hline
\textbf{Name} & \textbf{In} & \textbf{Out} & \textbf{Exceptions} \\
\hline
- & - & - & - \\
\hline
\end{tabular}
\end{center}

\subsection{Semantics}

This section provides the detailed functionality and behavior of the module’s
functions.

\subsubsection{State Variables}

None

\subsubsection{Environment Variables}

None

\subsubsection{Assumptions}

\begin{itemize}
\item a1
\end{itemize}

\subsubsection{Access Routine Semantics}

\noindent a:
\begin{itemize}
\item transition: 
\item output: 
\item exception: 
\end{itemize}

\subsubsection{Local Functions}

None

\newpage

\section{MIS of Logger Module} \label{MIS_Logger}

\subsection{Module}

Logger

\subsection{Uses}

\begin{itemize}
\item Hardware-Hiding Module
\end{itemize}

\subsection{Syntax}

This section describes the syntax used in the Logger Module, including the
constants, programs, and routines that can be accessed.

\subsubsection{Exported Constants}

None

\subsubsection{Exported Access Programs}

\begin{center}
\begin{tabular}{p{2cm} p{4cm} p{4cm} p{2cm}}
\hline
\textbf{Name} & \textbf{In} & \textbf{Out} & \textbf{Exceptions} \\
\hline
- & - & - & - \\
\hline
\end{tabular}
\end{center}

\subsection{Semantics}

This section provides the detailed functionality and behavior of the module’s
functions.

\subsubsection{State Variables}

None

\subsubsection{Environment Variables}

None

\subsubsection{Assumptions}

\begin{itemize}
\item a1
\end{itemize}

\subsubsection{Access Routine Semantics}

\noindent a:
\begin{itemize}
\item transition: 
\item output: 
\item exception: 
\end{itemize}

\subsubsection{Local Functions}

None

\newpage

\bibliographystyle {plainnat}
\bibliography {../../../refs/References}

\newpage

\section{Appendix} \label{Appendix}

\wss{Extra information if required}

\end{document}