\documentclass[12pt, titlepage]{article}

\usepackage{booktabs}
\usepackage{tabularx}
\usepackage{hyperref}
\hypersetup{ colorlinks, citecolor=black, filecolor=black, linkcolor=red,
    urlcolor=blue }
\usepackage[round]{natbib}

\input{../Comments}
\input{../Common}

\begin{document}

\title{Verification and Validation Report: \progname} 
\author{\authname}
\date{\today}
	
\maketitle

\pagenumbering{roman}

\section{Revision History}

\begin{tabularx}{\textwidth}{p{3cm}p{2cm}X} \toprule {\bf Date} & {\bf Version}
& {\bf Notes}\\
\midrule
April 11, 2025 & 1.0 & Creation\\
\bottomrule
\end{tabularx}

~\newpage

\section{Symbols, Abbreviations and Acronyms}

\renewcommand{\arraystretch}{1.2}
\begin{tabular}{l l} 
  \toprule		
  \textbf{symbol} & \textbf{description}\\
  \midrule 
  T & Test\\
  \bottomrule
\end{tabular}\\

\wss{symbols, abbreviations or acronyms -- you can reference the SRS tables if needed}

\newpage

\tableofcontents

\listoftables %if appropriate

\listoffigures %if appropriate

\newpage

\pagenumbering{arabic}

This document ...

\section{Functional Requirements Evaluation}

\section{Nonfunctional Requirements Evaluation}

\subsection{Usability}
		
\subsection{Performance}

\subsection{etc.}
	
\section{Comparison to Existing Implementation}	

This section will not be appropriate for every project.

\section{Unit Testing}

\section{Changes Due to Testing}

\wss{This section should highlight how feedback from the users and from the
supervisor (when one exists) shaped the final product.  In particular the
feedback from the Rev 0 demo to the supervisor (or to potential users) should be
highlighted.}

\section{Automated Testing}
		
\section{Trace to Requirements}
		
\section{Trace to Modules}		

\section{Code Coverage Metrics}

Code coverage for this project is evaluated using the \texttt{lcov} tool.
Automated coverage reports are generated and uploaded through GitHub Actions to
Codecov. The coverage dashboard can be accessed at:

\begin{center}
\url{https://app.codecov.io/gh/Lychee-acaca/CAS741}
\end{center}

As of the current assessment, the overall test coverage is \textbf{95.01\%}. The
coverage statistics for each major source file are summarized below:

\begin{center}
\begin{tabular}{|l|c|c|}
\hline
\textbf{File} & \textbf{Lines Covered / Total} & \textbf{Coverage (\%)} \\
\hline
\texttt{RwaveDetect.cpp}         & 15 / 15    & 100.00\% \\
\texttt{annRMSE.cpp}             & 18 / 19    & 94.74\%  \\
\texttt{annRMSE.hpp}             & 1 / 1      & 100.00\% \\
\texttt{dataStructure.hpp}       & 91 / 94    & 96.81\%  \\
\texttt{generalDigitalFilter.cpp}& 24 / 24    & 100.00\% \\
\texttt{generalDigitalFilter.hpp}& 10 / 10    & 100.00\% \\
\texttt{io\_processing.cpp}       & 49 / 55    & 89.09\%  \\
\texttt{io\_processing.hpp}       & 3 / 3      & 100.00\% \\
\texttt{logger.cpp}              & 33 / 37    & 89.19\%  \\
\texttt{logger.hpp}              & 12 / 12    & 100.00\% \\
\texttt{mmath.cpp}               & 50 / 52    & 96.15\%  \\
\texttt{pantomp.cpp}             & 127 / 134  & 94.78\%  \\
\texttt{pantomp.hpp}             & 5 / 5      & 100.00\% \\
\hline
\end{tabular}
\end{center}

This high level of coverage demonstrates that most of the critical logic in the
project has been thoroughly tested.

\bibliographystyle{plainnat}
\bibliography{../../refs/References}

\newpage{}
\section*{Appendix --- Reflection}

The information in this section will be used to evaluate the team members on the
graduate attribute of Reflection.

\input{../Reflection.tex}

\begin{enumerate}
  \item What went well while writing this deliverable? 
  \item What pain points did you experience during this deliverable, and how did
    you resolve them?
  \item Which parts of this document stemmed from speaking to your client(s) or
  a proxy (e.g. your peers)? Which ones were not, and why?
  \item In what ways was the Verification and Validation (VnV) Plan different
  from the activities that were actually conducted for VnV?  If there were
  differences, what changes required the modification in the plan?  Why did
  these changes occur?  Would you be able to anticipate these changes in future
  projects?  If there weren't any differences, how was your team able to clearly
  predict a feasible amount of effort and the right tasks needed to build the
  evidence that demonstrates the required quality?  (It is expected that most
  teams will have had to deviate from their original VnV Plan.)
\end{enumerate}

\end{document}