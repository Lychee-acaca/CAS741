% THIS DOCUMENT IS TAILORED TO REQUIREMENTS FOR SCIENTIFIC COMPUTING.  IT
% SHOULDN'T BE USED FOR NON-SCIENTIFIC COMPUTING PROJECTS
\documentclass[12pt]{article}

\usepackage{amsmath, mathtools}
\usepackage{amsfonts}
\usepackage{amssymb}
\usepackage{graphicx}
\usepackage{colortbl}
\usepackage{xr}
\usepackage{hyperref}
\usepackage{longtable}
\usepackage{xfrac}
\usepackage{tabularx}
\usepackage{float}
\usepackage{siunitx}
\usepackage{booktabs}
\usepackage{caption}
\usepackage{pdflscape}
\usepackage{afterpage}
\usepackage{cite}

%\usepackage[round]{natbib}

%\usepackage{refcheck}

\hypersetup{ bookmarks=true,         % show bookmarks bar?
      colorlinks=true,       % false: boxed links; true: colored links
    linkcolor=red,          % color of internal links (change box color with
                            % linkbordercolor)
    citecolor=green,        % color of links to bibliography
    filecolor=magenta,      % color of file links
    urlcolor=cyan           % color of external links
}

\input{../Comments}
\input{../Common}

% For easy change of table widths
\newcommand{\colZwidth}{1.0\textwidth}
\newcommand{\colAwidth}{0.13\textwidth}
\newcommand{\colBwidth}{0.82\textwidth}
\newcommand{\colCwidth}{0.1\textwidth}
\newcommand{\colDwidth}{0.05\textwidth}
\newcommand{\colEwidth}{0.8\textwidth}
\newcommand{\colFwidth}{0.17\textwidth}
\newcommand{\colGwidth}{0.5\textwidth}
\newcommand{\colHwidth}{0.28\textwidth}

% Used so that cross-references have a meaningful prefix
\newcounter{defnum} %Definition Number
\newcommand{\dthedefnum}{GD\thedefnum}
\newcommand{\dref}[1]{GD\ref{#1}} \newcounter{datadefnum} %Datadefinition Number
\newcommand{\ddthedatadefnum}{DD\thedatadefnum}
\newcommand{\ddref}[1]{DD\ref{#1}} \newcounter{theorynum} %Theory Number
\newcommand{\tthetheorynum}{TM\thetheorynum}
\newcommand{\tref}[1]{TM\ref{#1}} \newcounter{tablenum} %Table Number
\newcommand{\tbthetablenum}{TB\thetablenum}
\newcommand{\tbref}[1]{TB\ref{#1}} \newcounter{assumpnum} %Assumption Number
\newcommand{\atheassumpnum}{A\theassumpnum}
\newcommand{\aref}[1]{A\ref{#1}} \newcounter{goalnum} %Goal Number
\newcommand{\gthegoalnum}{GS\thegoalnum}
\newcommand{\gsref}[1]{GS\ref{#1}} \newcounter{instnum} %Instance Number
\newcommand{\itheinstnum}{IM\theinstnum}
\newcommand{\iref}[1]{IM\ref{#1}} \newcounter{reqnum} %Requirement Number
\newcommand{\rthereqnum}{R\thereqnum}
\newcommand{\rref}[1]{R\ref{#1}} \newcounter{nfrnum} %NFR Number
\newcommand{\rthenfrnum}{NFR\thenfrnum}
\newcommand{\nfrref}[1]{NFR\ref{#1}} \newcounter{lcnum} %Likely change number
\newcommand{\lthelcnum}{LC\thelcnum}
\newcommand{\lcref}[1]{LC\ref{#1}}

\usepackage{fullpage}

\newcommand{\deftheory}[9][Not Applicable]
{
\newpage
\noindent \rule{\textwidth}{0.5mm}

\paragraph{RefName: } \textbf{#2} \phantomsection 

\paragraph{Label:} #3
\label{#3}

\noindent \rule{\textwidth}{0.5mm}

\paragraph{Equation:}

#4

\paragraph{Description:}

#5

\paragraph{Notes:}

#6

\paragraph{Source:}

#7

\paragraph{Ref.\ By:}

#8

\paragraph{Preconditions for \hyperref[#2]{#2}:}
\label{#2_precond}

#9

\paragraph{Derivation for \hyperref[#2]{#2}:}
\label{#2_deriv}

#1

\noindent \rule{\textwidth}{0.5mm}

}

\begin{document}

\title{Software Requirements Specification for \progname: subtitle describing software} 
\author{\authname}
\date{\today}
	
\maketitle

~\newpage

\pagenumbering{roman}

\tableofcontents

~\newpage

\section*{Revision History}

\begin{tabularx}{\textwidth}{p{3cm}p{2cm}X} \toprule {\bf Date} & {\bf Version}
& {\bf Notes}\\
\midrule
January 16, 2025 & 1.0 & Creation\\
\bottomrule
\end{tabularx}

~\\
\plt{This template is intended for use by CAS 741.  For CAS 741 the template
should be used exactly as given, except the Reflection Appendix can be deleted.
For the capstone course it is a source of ideas, but shouldn't be followed
exactly.  The exception is the reflection appendix.  All capstone SRS documents
should have a reflection appendix.}

~\newpage

\section{Reference Material}

This section records information for easy reference.

\subsection{Table of Units}

Throughout this document SI (Syst\`{e}me International d'Unit\'{e}s) is employed
as the unit system.  In addition to the basic units, several derived units are
used as described below.  For each unit, the symbol is given followed by a
description of the unit and the SI name.  ~\newline

\renewcommand{\arraystretch}{1.2}
%\begin{table}[ht]
  \noindent \begin{tabular}{l l l} 
    \toprule		
    \textbf{symbol} & \textbf{unit} & \textbf{SI}\\
    \midrule 
    \si{\second} & time & second\\
    \si{\volt} & voltage & volt\\
    \si{\hertz} & frequency & hertz (\si{\hertz} = \si{\per\second})\\
    \bottomrule
  \end{tabular}
  %	\caption{Provide a caption}
%\end{table}

\plt{Only include the units that your SRS actually uses.}

\plt{Derived units, like newtons, pascal, etc, should show their derivation (the
    units they are derived from) if their constituent units are in the table of
    units (that is, if the units they are derived from are used in the
    document).  For instance, the derivation of pascals as
    $\si{\pascal}=\si{\newton\per\square\meter}$ is shown if newtons and m are
    both in the table.  The derivations of newtons would not be shown if kg and
    s are not both in the table.}

\plt{The symbol for units named after people use capital letters, but the name
  of the unit itself uses lower case.  For instance, pascals use the symbol Pa,
  watts use the symbol W, teslas use the symbol T, newtons use the symbol N,
  etc.  The one exception to this is degree Celsius.  Details on writing metric
  units can be found on the 
  \href{https://www.nist.gov/pml/weights-and-measures/writing-metric-units}
  {NIST} web-page.}

\noindent Exceptionally, some units officially accepted for use with the SI are
used as described below.  ~\newline

\renewcommand{\arraystretch}{1.2}
%\begin{table}[ht]
  \noindent \begin{tabular}{l l} 
    \toprule		
    \textbf{symbol} & \textbf{unit}\\
    \midrule 
    \si{\decibel} & logarithmic ratio quantity\\
    \bottomrule
  \end{tabular}
  %	\caption{Provide a caption}
%\end{table}

\subsection{Table of Symbols}

The table that follows summarizes the symbols used in this document along with
their units.  The choice of symbols was made to be consistent with the existing
documentation for the R-wave detector.  The symbols are listed in alphabetical
order.

\renewcommand{\arraystretch}{1.2}
%\noindent \begin{tabularx}{1.0\textwidth}{l l X}
\noindent \begin{longtable*}{l l p{12cm}} \toprule \textbf{symbol} &
\textbf{unit} & \textbf{description}\\
\midrule 
$a$ & N/A & a sequence of digital filter weighting coefficients \\
$A$ & \si{\mV} & a sequence of index of the annotated R wave \\
$b$ & N/A & a sequence of digital filter feedback coefficients \\
$f_c$ & \si{\hertz} & cutoff frequency of the filter \\
$f_s$ & \si{\hertz} & sampling frequency of the ECG signal \\
$H$ & N/A & transfer function of digital filter \\
$R$ & N/A & a sequence of index of the calculated R wave \\
$u$ & \si{\mV} & a sequence of discrete-time ECG signal \\
$z$ & N/A & complex frequency variable in the Z-domain, $z=e^{j2\pi f}$ \\
\bottomrule
\end{longtable*}
\plt{Use your problems actual symbols.  The si package is a good idea to use for
  units.}

\subsection{Abbreviations and Acronyms}

\renewcommand{\arraystretch}{1.2}
\begin{tabular}{l l} 
  \toprule		
  \textbf{symbol} & \textbf{description}\\
  \midrule 
  A & Assumption\\
  DD & Data Definition\\
  GD & General Definition\\
  GS & Goal Statement\\
  IM & Instance Model\\
  LC & Likely Change\\
  PS & Physical System Description\\
  R & Requirement\\
  SRS & Software Requirements Specification\\
  \progname{} & R-wave Detection\\
  TM & Theoretical Model\\
  ECG & Electrocardiography\\
  ADC & Analog-to-Digital Converter \\
  RMSE & Root Mean Square Error\\
  \bottomrule
\end{tabular}\\

\plt{Add any other abbreviations or acronyms that you add}

\subsection{Mathematical Notation}

In the fields of electronics and signal processing, $j$ represents the imaginary
unit. In other word, we define $j^2 = -1$.

In discrete signal processing, square brackets are used to represent discrete
signals and round brackets are used to represent continuous signals.  For
instance, $u(t)$ is continuous signal and $u[n]$ is discrete signal.

\newpage

\pagenumbering{arabic}

\plt{This SRS template is based on \cite{SmithAndLai2005, SmithEtAl2007,
  SmithAndKoothoor2016}.  It will get you started.  You should not modify the
  section headings, without first discussing the change with the course
  instructor.  Modification means you are not following the template, which
  loses some of the advantage of a template, especially standardization.
  Although the bits shown below do not include type information, you may need to
  add this information for your problem.  If you are unsure, please can ask the
  instructor.}

\plt{Feel free to change the appearance of the report by modifying the LaTeX
  commands.}

\plt{This template document assumes that a single program is being documented.
  If you are documenting a family of models, you should start with a commonality
  analysis.  A separate template is provided for this.  For program families you
  should look at \cite{Smith2006, SmithMcCutchanAndCarette2017}.  Single family
  member programs are often programs based on a single physical model.  General
  purpose tools are usually documented as a family.  Families of physical models
  also come up.}

\plt{The SRS is not generally written, or read, sequentially.  The SRS is a
  reference document.  It is generally read in an ad hoc order, as the need
  arises.  For writing an SRS, and for reading one for the first time, the
  suggested order of sections is:
\begin{itemize}
\item Goal Statement
\item Instance Models
\item Requirements
\item Introduction
\item Specific System Description
\end{itemize}
}

\plt{Guiding principles for the SRS document:
\begin{itemize}
\item Do not repeat the same information at the same abstraction level.  If
  information is repeated, the repetition should be at a different abstraction
  level.  For instance, there will be overlap between the scope section and the
  assumptions, but the scope section will not go into as much detail as the
  assumptions section.
\end{itemize}
}

\plt{The template description comments should be disabled before submitting this
  document for grading.}

\plt{You can borrow any wording from the text given in the template.  It is part
  of the template, and not considered an instance of academic integrity.  Of
  course, you need to cite the source of the template.}

\plt{When the documentation is done, it should be possible to trace back to the
  source of every piece of information.  Some information will come from
  external sources, like terminology.  Other information will be derived, like
  General Definitions.}

\plt{An SRS document should have the following qualities: unambiguous,
  consistent, complete, validatable, abstract and traceable.}

\plt{The overall goal of the SRS is that someone that meets the Characteristics
  of the Intended Reader (Section~\ref{sec_IntendedReader}) can learn,
  understand and verify the captured domain knowledge.  They should not have to
  trust the authors of the SRS on any statements.  They should be able to
  independently verify/derive every statement made.}

\section{Introduction}

R-wave detection is a critical task in ECG signal processing, serving important
purposes such as heart rate calculation, arrhythmia analysis, cardiac conduction
evaluation, and heart disease diagnosis.  The patient's ECG signal is sampled by
the sensor and converted into a digital signal, where we can detect R-waves.
However, since the sampling is not under an ideal condition, a large amount of
clutter wave is mixed into the original signal, such as electrical noises and
utility frequency.

There are many algorithms for detecting R-waves.  This project will focus on
reimplementing the Pan-Tompkins\cite{4122029} algorithm.  All filter parameters
used in the algorithm will be automatically derived according to requirements
without relying on external libraries.

\plt{The introduction section is written to introduce the problem.  It starts
  general and focuses on the problem domain. The general advice is to start with
  a paragraph or two that describes the problem, followed by a ``roadmap''
  paragraph.  A roadmap orients the reader by telling them what sub-sections to
  expect in the Introduction section.}

\subsection{Purpose of Document}

The primary purpose of this document is to record the requirements of \progname.
Goals, assumptions, theoretical models, definitions, and other model derivation
information are specified, allowing the reader to fully understand and verify
the purpose and scientific basis of \progname.  With the exception of system
constraints, this SRS will remain abstract, describing what problem is being
solved, but not how to solve it.

This document will be used as a starting point for subsequent development
phases, including writing the design specification and the software verification
and validation plan.  The design document will show how the requirements are to
be realized, including decisions on the numerical algorithms and programming
environment.  The verification and validation plan will show the steps that will
be used to increase confidence in the software documentation and the
implementation.  Although the SRS fits in a series of documents that follow the
so-called waterfall model, the actual development process is not constrained in
any way.

\subsection{Scope of Requirements} 

The scope of the requirements includes processing sampled data from a single
channel.

\subsection{Characteristics of Intended Reader} \label{sec_IntendedReader}

The intended readers of this document include individuals with a background in
signal processing, biomedical engineering, and software development,
particularly those working with digital filters and ECG signal analysis.

\subsection{Organization of Document}

\plt{This section provides a roadmap of the SRS document.  It will help the
  reader orient themselves.  It will provide direction that will help them
  select which sections they want to read, and in what order.  This section will
  be similar between project.}

\section{General System Description}

This section provides general information about the system.  It identifies the
interfaces between the system and its environment, describes the user
characteristics and lists the system constraints.  \plt{This text can likely be
borrowed verbatim.}

\plt{The purpose of this section is to provide general information about the
  system so the specific requirements in the next section will be easier to
  understand. The general system description section is designed to be
  changeable independent of changes to the functional requirements documented in
  the specific system description. The general system description provides a
  context for a family of related models.  The general description can stay the
  same, while specific details are changed between family members.}

\subsection{System Context}

The RwaveDetecion system receives sample frequency $f_s$ and discrete time
domain ECG signal $u$ from the ADC.  Based on this information, the index of
each R-wave peak $R$ can be calculated and returned to the user.  If the user
provides annotated data $A$, which is optional, the system will return the
$RMSE$ between $A$ and $R$ additionally.

\plt{Your system context will include a figure that shows the abstract view of
  the software.  Often in a scientific context, the program can be viewed
  abstractly following the design pattern of Inputs $\rightarrow$ Calculations
  $\rightarrow$ Outputs.  The system context will therefore often follow this
  pattern.  The user provides inputs, the system does the calculations, and then
  provides the outputs to the user.  The figure should not show all of the
  inputs, just an abstract view of the main categories of inputs (like material
  properties, geometry, etc.).  Likewise, the outputs should be presented from
  an abstract point of view.  In some cases the diagram will show other external
  entities, besides the user.  For instance, when the software product is a
  library, the user will be another software program, not an actual end user. If
  there are system constraints that the software must work with external
  libraries, these libraries can also be shown on the System Context diagram.
  They should only be named with a specific library name if this is required by
  the system constraint.}

\begin{figure}[h!]
\begin{center}
 \includegraphics[width=0.8\textwidth]{SystemContextFigure}
\caption{System Context}
\label{Fig_SystemContext} 
\end{center}
\end{figure}

\plt{For each of the entities in the system context diagram its responsibilities
  should be listed.  Whenever possible the system should check for data quality,
  but for some cases the user will need to assume that responsibility.  The list
  of responsibilities should be about the inputs and outputs only, and they
  should be abstract.  Details should not be presented here.  However, the
  information should not be so abstract as to just say ``inputs'' and
  ``outputs''.  A summarizing phrase can be used to characterize the inputs. For
  instance, saying ``material properties'' provides some information, but it
  stays away from the detail of listing every required properties.}

\begin{itemize}
\item User Responsibilities:
\begin{itemize}
\item Provide integer ADC sample frequency $f_s$
\item Provide discrete time domain ECG signal real sequence $u$ sampled from ADC
\item (Optional) Provide correctly annotated data real sequence $A$
\end{itemize}
\item \progname{} Responsibilities:
\begin{itemize}
\item Detect data type mismatch, such as a string of characters instead of a
  real number
\item Return the integer sequence index of each R-wave peak $R$ detected from
$u$
\item Render a graph that shows $u$ and $R$ at the same time
\item Return the $RMSE$ between $A$ and $R$ if required
\end{itemize}
\end{itemize}

\plt{Identify in what context the software will typically be used.  Is it for
exploration? education? engineering work? scientific work?. Identify whether it
will be used for mission-critical or safety-critical applications.} \plt{This
additional context information is needed to determine how much effort should be
devoted to the rationale section.  If the application is safety-critical, the
bar is higher.  This is currently less structured, but analogous to, the idea to
the Automotive Safety Integrity Levels (ASILs) that McSCert uses in their
automotive hazard analyses.}

\subsection{User Characteristics} \label{SecUserCharacteristics}

The end user of RwaveDetection needs to have a basic understanding of the QRS
complex in ECG.  Advanced user who hopes to evaluate the accuracy of the system
should have an understanding of undergraduate level mathematics.

\plt{This section summarizes the knowledge/skills expected of the user.
  Measuring usability, which is often a required non-function requirement,
  requires knowledge of a typical user.  As mentioned above, the user is a
  different role from the ``intended reader,'' as given in
  Section~\ref{sec_IntendedReader}.  As in Section~\ref{sec_IntendedReader}, the
  user characteristics should be specific an unambiguous.  For instance, ``The
  end user of \progname{} should have an understanding of undergraduate Level 1
  Calculus and Physics.''}

\subsection{System Constraints}

RwaveDetection should be able to process data in real time and it can be used in
various low performance embedded devices, so it requires a Low-Level programming
language to ensure efficiency and compatibility.

\plt{System constraints differ from other type of requirements because they
  limit the developers' options in the system design and they identify how the
  eventual system must fit into the world. This is the only place in the SRS
  where design decisions can be specified.  That is, the quality requirement for
  abstraction is relaxed here.  However, system constraints should only be
  included if they are truly required.}

\section{Specific System Description}

This section first presents the problem description, which gives a high-level
view of the problem to be solved.  This is followed by the solution
characteristics specification, which presents the assumptions, theories,
definitions and finally the instance models.  \plt{Add any project specific
details that are relevant for the section overview.}

\subsection{Problem Description} \label{Sec_pd}

\progname{} is intended to find the R-wave peak position accurately from single
channel ECG data containing clutter and noise.

\plt{What problem does your program solve?  The description here should be in
the problem space, not the solution space.}

\subsubsection{Terminology and  Definitions}

\plt{This section is expressed in words, not with equations.  It provide the
  meaning of the different words and phrases used in the domain of the problem.
  The terminology is used to introduce concepts from the world outside of the
  mathematical model  The terminology provides a real world connection to give
  the mathematical model meaning.}

This subsection provides a list of terms that are used in the subsequent
sections and their meaning, with the purpose of reducing ambiguity and making it
easier to correctly understand the requirements:

\begin{itemize}

\item QRS complex: the combination of three of the graphical deflections seen on
a typical electrocardiogram (ECG or EKG)\cite{wiki:QRS_complex}
\item R-wave: represents the peak of ventricular depolarization
\item Filter: a device or process that removes some unwanted components or
features from a signal\cite{wiki:Filter_(signal_processing)}
\item Utility frequency: the nominal frequency of the oscillations of
alternating current (AC) in a wide area synchronous grid transmitted from a
power station to the end-user.\cite{wiki:Utility_frequency}

\end{itemize}

\subsubsection{Physical System Description} \label{sec_phySystDescrip}

\plt{The purpose of this section is to clearly and unambiguously state the
  physical system that is to be modelled. Effective problem solving requires a
  logical and organized approach. The statements on the physical system to be
  studied should cover enough information to solve the problem. The physical
  description involves element identification, where elements are defined as
  independent and separable items of the physical system. Some example elements
  include acceleration due to gravity, the mass of an object, and the size and
  shape of an object. Each element should be identified and labelled, with their
  interesting properties specified clearly. The physical description can also
  include interactions of the elements, such as the following: i) the
  interactions between the elements and their physical environment; ii) the
  interactions between elements; and, iii) the initial or boundary conditions.}

\plt{The elements of the physical system do not have to correspond to an actual
physical entity.  They can be conceptual.  This is particularly important when
the documentation is for a numerical method. }

The physical system of \progname{}, as shown in Figure~?, includes the following
elements:

\begin{itemize}

\item[PS1:] The heart generates electrical impulses that propagate through the
body, causing voltage differences at the skin surface.

\item[PS2:] Electrodes placed on the body detect the electrical signals from the
heart.  These signals are captured in the form of voltage differences between
the electrode pairs.

\item[PS3:] Noise elements such as muscle artifacts, motion artifacts, and
electrical interference from external sources (e.g., power lines) can corrupt
the ECG signal.

\end{itemize}

\plt{A figure here makes sense for most SRS documents}

% \begin{figure}[h!] \begin{center} %\rotatebox{-90}
% {
%  \includegraphics[width=0.5\textwidth]{<FigureName>}
% }
% \caption{\label{<Label>} <Caption>} \end{center} \end{figure}

\subsubsection{Goal Statements}

\begin{itemize}

\item[GS\refstepcounter{goalnum}\thegoalnum \label{G_find_index}:] Given a
single-channel unfiltered ECG signal, find the index of each R-wave peak.

\item[GS\refstepcounter{goalnum}\thegoalnum \label{G_calculate_RMSE}:] Given
correct annotated data, calculate the RMSE between each detected R-wave peak
time and annotated time.

\end{itemize}

\subsection{Solution Characteristics Specification}

% \plt{This section specifies the information in the solution domain of the
%   system to be developed. This section is intended to express what is required
%   in such a way that analysts and stakeholders get a clear picture, and the
%   latter will accept it. The purpose of this section is to reduce the problem
%   into one expressed in mathematical terms. Mathematical expertise is used to
%   extract the essentials from the underlying physical description of the
%   problem, and to collect and substantiate all physical data pertinent to the
%   problem.}

% \plt{This section presents the solution characteristics by successively
%   refining models.  It starts with the abstract/general Theoretical Models
%   (TMs) and refines them to the concrete/specific Instance Models (IMs).  If
%   necessary there are intermediate refinements to General Definitions (GDs).
%   All of these refinements can potentially use Assumptions (A) and Data
%   Definitions (DD). TMs are refined to create new models, that are called GMs
%   or IMs. DDs are not refined; they are just used. GDs and IMs are derived, or
%   refined, from other models. DDs are not derived; they are just given. TMs
%   are also just given, but they are refined, not used.  If a potential DD
%   includes a derivation, then that means it is refining other models, which
%   would make it a GD or an IM.}

% \plt{The above makes a distinction between ``refined'' and ``used.'' A model
%   is refined to another model if it is changed by the refinement. When we
%   change a general 3D equation to a 2D equation, we are making a refinement,
%   by applying the assumption that the third dimension does not matter. If we
%   use a definition, like the definition of density, we aren't refining, or
%   changing that definition, we are just using it.}

% \plt{The same information can be a TM in one problem and a DD in another.  It
%   is about how the information is used.  In one problem the definition of
%   acceleration can be a TM, in another it would be a DD.}

% \plt{There is repetition between the information given in the different chunks
%   (TM, GDs etc) with other information in the document.  For instance, the
%   meaning of the symbols, the units etc are repeated.  This is so that the
%   chunks can stand on their own when being read by a reviewer/user.  It also
%   facilitates reuse of the models in a different context.}

% \noindent \plt{The relationships between the parts of the document are show in
%   the following figure.  In this diagram ``may ref'' has the same role as
%   ``uses'' above.  The figure adds ``Likely Changes,'' which are able to
%   reference (use) Assumptions.}

\begin{figure}[H]
  \includegraphics[scale=0.9]{RelationsBetweenTM_GD_IM_DD_A.pdf}
\end{figure}

The instance models that govern \progname{} are presented in
Subsection~\ref{sec_instance}.  The information to understand the meaning of the
instance models and their derivation is also presented, so that the instance
models can be verified.

% \subsubsection{Types}

% \plt{This section is optional. Defining types can make the document easier to
% understand.}

\subsubsection{Assumptions} \label{sec_assumpt}

\plt{The assumptions are a refinement of the scope.  The scope is general, where
  the assumptions are specific.  All assumptions should be listed, even those
  that domain experts know so well that they are rarely (if ever) written down.}
  \plt{The document should not take for granted that the reader knows which
  assumptions have been made. In the case of unusual assumptions, it is
  recommended that the documentation either include, or point to, an explanation
  and justification for the assumption.} \plt{If it helps with the organization
  and understandability, the assumptions can be presented as sub sections.  The
  following sub-sections are options: background theory assumptions, helper
  theory assumptions, generic theory assumptions, problem specific assumptions,
  and rationale assumptions}

This section simplifies the original problem and helps in developing the
theoretical model by filling in the missing information for the physical system.
The numbers given in the square brackets refer to the theoretical model [TM],
general definition [GD], data definition [DD], instance model [IM], or likely
change [LC], in which the respective assumption is used.

\begin{itemize}

\item[A\refstepcounter{assumpnum}\theassumpnum \label{A_Nq_frequency}:] The
sampling frequency is not lower than the Nyquist frequency.

\item[A\refstepcounter{assumpnum}\theassumpnum \label{A_filtered_out}:] Noise in
data can be filtered out, within expectations.

\item[A\refstepcounter{assumpnum}\theassumpnum \label{A_naked_eye}:] The R-wave
has peak characteristics that can be discerned by the naked eye.

\item[A\refstepcounter{assumpnum}\theassumpnum \label{A_high_quality}:] The
input ECG signal is of high quality.

\end{itemize}

\subsubsection{Theoretical Models}\label{sec_theoretical}

~\newline
\refstepcounter{theorynum}
\noindent
\deftheory
% #2 refname of theory
{TM\thetheorynum}
% #3 label
{Analog Filter Equation}
% #4 equation
{$H(s) = \frac{\sum_{i=0}^{m}{b_is^i}}{\sum_{i=0}^{n}{a_is^i}}$}
% #5 description
{ \leavevmode \\
  $H$ is the transfer function of the filter.  \\
  $s$ is the Laplace variable (complex frequency).  \\
  $b_m$, $b_{m-1}$, $...$, $b_0$ are the numerator coefficients.  \\
  $a_n$, $a_{n-1}$, $...$, $a_0$ are the denominator coefficients, where $a_n
  \neq 0$ to ensure causality and stability.  \\
  $n$ and $m$ represent the highest order of the denominator and numerator
  polynomials, respectively.  Usually, $n \geq m$ to maintain realizability
  (strictly causal system).  }
% #6 Notes
{ Transfer functions are commonly used in the analysis of systems such as
  single-input single-output filters in signal processing, communication theory,
  and control theory.  }
% #7 Source
{
  \url{https://en.wikipedia.org/wiki/Transfer_function}
}
% #8 Referenced by
{
  \dref{ROCT}
}
% #9 Preconditions
{ None }
% #1 derivation - not applicable by default
{}

~\newline
\refstepcounter{theorynum}
\noindent
\deftheory
% #2 refname of theory
{TM\thetheorynum}
% #3 label
{Butterworth Filter}
% #4 equation
{$G_n(\omega) = \frac{G_0}{\sqrt{1+(\frac{\omega}{\omega_c})^{2n}}}$}
% #5 description
{ \leavevmode \\
  $G_n$ is the gain function of the $n$th-order Butterworth filter.  \\
  $G_0$ is the DC gain (gain at zero frequency).  \\
  $\omega$ is the angular frequency of the signal.  \\
  $\omega_c$ is the angular cutoff frequency (\si{\hertz}) which defines the
  $-3\si{\decibel}$ point.  \\
  $n$ is the order of the filter.  }
% #6 Notes
{The frequency response of the Butterworth filter is maximally flat (i.e., has
no ripples) in the passband and rolls off towards zero in the stopband.  }
% #7 Source
{
  \url{https://en.wikipedia.org/wiki/Butterworth_filter}
}
% #8 Referenced by
{
  \dref{ROCT}
}
% #9 Preconditions
{ None }
% #1 derivation - not applicable by default
{}

~\newline
\refstepcounter{theorynum}
\noindent
\deftheory
% #2 refname of theory
{TM\thetheorynum}
% #3 label
{Chebyshev Filter}
% #4 equation
{$G_n(\omega) = \frac{1}{\sqrt{1+\epsilon^2T_n^2(\omega/\omega_c)}}$}
% #5 description
{ \leavevmode \\
$G_n$ is the gain function of the $n$th-order Butterworth filter.  \\
$\epsilon$ is the ripple factor, $|\epsilon| < 1$ \\
$\omega$ is the angular frequency of the signal.  \\
$\omega_c$ is the angular cutoff frequency (\si{\hertz}) which defines the
$-3\si{\decibel}$ point.  \\
$T_n(\frac{\omega}{\omega0})$ is the $n$th order Chebyshev polynomials.  
}
% #6 Notes
{Chebyshev filters have a steeper roll-off than Butterworth filters, and have
either passband ripple (type I) or stopband ripple (type II).  }
% #7 Source
{
  \url{https://en.wikipedia.org/wiki/Chebyshev_filter}
}
% #8 Referenced by
{
  \dref{ROCT}
}
% #9 Preconditions
{ None }
% #1 derivation - not applicable by default
{}

~\newline
\refstepcounter{theorynum}
\noindent
\deftheory
% #2 refname of theory
{TM\thetheorynum}
% #3 label
{Threshold Detect}
% #4 equation
{ $T[n] = 
\begin{cases}
x[n] & \text{if } x[n] \geq \theta \\
0 & \text{if } x[n] < \theta \end{cases}$ }
% #5 description
{ \leavevmode \\
  $x$ is the input signal.  \\
  $T$ is the output signal.  \\
  $\theta$ is the threshold, which is not guaranteed to be a static number all
  the time and may change as the state changes.  \\
}
% #6 Notes
{threshold comparison detection is a simple signal processing technique where
the input signal is compared to a threshold value.}
% #7 Source
{}
% #8 Referenced by
{
  \dref{ROCT}
}
% #9 Preconditions
{ None }
% #1 derivation - not applicable by default
{}

~\newpage

\subsubsection{General Definitions}\label{sec_gendef}

This section collects the laws and equations that will be used in building the
instance models.

~\newline

\noindent
\begin{minipage}{\textwidth}
\renewcommand*{\arraystretch}{1.5}
\begin{tabular}{| p{\colAwidth} | p{\colBwidth}|}
\hline
\rowcolor[gray]{0.9}
Number& GD\refstepcounter{defnum}\thedefnum \label{B-trans}\\
\hline
Label &\bf Bilinear transform \\
\hline
% Units&$MLt^{-3}T^0$\\
% \hline SI Units&\si{\watt\per\square\metre}\\
% \hline
Equation&$s \approx \frac{2}{T} \cdot \frac{z-1}{z+1}$  \\
\hline
Description & The bilinear transform is used in digital signal processing and
discrete-time control theory to transform continuous-time system representations
to discrete-time.  \\
& $s$ is the Laplace variable (complex frequency).  \\
& $z$ is the complex frequency variable in the Z-domain.  \\
& $T$ is the sampling period.  \\
& When \aref{A_Nq_frequency} is satisfied, this equation represents replacing
all $s$ terms in $H(s)$ S-domain transfer function directly with $\frac{2}{T}
\cdot \frac{z-1}{z+1}$, after doing this we can get Z-domain transfer function
H(z).  \\
& Analog filters in \tref{Butterworth Filter} and \tref{Chebyshev Filter} can be
transformed into digital filter in this way. \\
\hline
  Source & \url{https://en.wikipedia.org/wiki/Bilinear_transform} \\
  \hline
  Ref.\ By & \ddref{FluxCoil}, \ddref{FluxPCM}\\
  \hline
\end{tabular}
\end{minipage}\\

~\newline

\noindent
\begin{minipage}{\textwidth}
\renewcommand*{\arraystretch}{1.5}
\begin{tabular}{| p{\colAwidth} | p{\colBwidth}|}
\hline
\rowcolor[gray]{0.9}
Number& GD\refstepcounter{defnum}\thedefnum \label{FDE}\\
\hline
Label &\bf Filter Difference Equation \\
\hline
% Units&$MLt^{-3}T^0$\\
% \hline SI Units&\si{\watt\per\square\metre}\\
% \hline
Equation&$H(z)=\frac{\sum_{l=0}^{N}b_lz^{-l}}{1 + \sum_{k=1}^{M}a_kz^{-k}}
\leftrightarrow y[n] = -\sum_{k=1}^{M}a_ky[n-k] + \sum_{l=0}^{N}b_lx[n-l]$  \\
\hline
Description & $x[n]$ is the input signal at the $n$-th time sample.  \\
& $y[n]$ is the output signal at the $n$-th time sample.  \\
& $b_l$ is the coefficients for the input signal.  \\
& $a_k$ is the coefficients for the output signal.  \\
& $N$ is the maximum delay for the input signal.  \\
& $M$ is the maximum delay for the output signal.  \\
& $z$ is the complex frequency variable in the Z-domain.  \\
& This equation reflects the conversion relationship between the difference
equation and the Z-domain transfer function $H(z)$ comes from \dref{B-trans}. \\

\hline
  Source & \url{https://en.wikipedia.org/wiki/Digital_filter} \\
  \hline
  Ref.\ By & \ddref{FluxCoil}, \ddref{FluxPCM}\\
  \hline
\end{tabular}
\end{minipage}\\

~\newline

\noindent
\begin{minipage}{\textwidth}
\renewcommand*{\arraystretch}{1.5}
\begin{tabular}{| p{\colAwidth} | p{\colBwidth}|}
\hline
\rowcolor[gray]{0.9}
Number& GD\refstepcounter{defnum}\thedefnum \label{Differential filter}\\
\hline
Label &\bf Differential filter \\
\hline
% Units&$MLt^{-3}T^0$\\
% \hline SI Units&\si{\watt\per\square\metre}\\
% \hline
Equation&$y[n]=\frac{1}{8}(2x[n]+x[n-1]-x[n-3]-2x[n-4])$  \\
\hline
Description & $x[n]$ is the input signal at the $n$-th time sample.  \\
& $y[n]$ is the output signal at the $n$-th time sample.  \\
& The five-point differentiation formula is a numerical method used to
approximate the derivative of a discrete signal.  This filter is actually a
special digital filter using difference equations in \dref{FDE}.  \\
& In this program, differential filter can be used to enhance the slope of the
QRS complex.  \\
\hline
  Source & \url{https://en.wikipedia.org/wiki/Digital_filter} \\
  \hline
  Ref.\ By & \ddref{FluxCoil}, \ddref{FluxPCM}\\
  \hline
\end{tabular}
\end{minipage}\\


% \subsubsection*{Detailed derivation of simplified rate of change of
% temperature}

% \plt{This may be necessary when the necessary information does not fit in the
%   description field.} \plt{Derivations are important for justifying a given
%   GD.  You want it to be clear where the equation came from.}

\subsubsection{Data Definitions}\label{sec_datadef}

\plt{The Data Definitions are definitions of symbols and equations that are
  given for the problem.  They are not derived; they are simply used by other
  models.  For instance, if a problem depends on density, there may be a data
  definition for the equation defining density.  The DDs are given information
  that you can use in your other modules.}

\plt{All Data Definitions should be used (referenced) by at least one other
  model.}

This section collects and defines all the data needed to build the instance
models. The dimension of each quantity is also given.  \plt{Modify the examples
below for your problem, and add additional definitions as appropriate.}

~\newline

\noindent
\begin{minipage}{\textwidth}
\renewcommand*{\arraystretch}{1.5}
\begin{tabular}{| p{\colAwidth} | p{\colBwidth}|}
\hline
\rowcolor[gray]{0.9}
Number& DD\refstepcounter{datadefnum}\thedatadefnum \label{FluxCoil}\\
\hline
Label& \bf Heat flux out of coil\\
\hline
Symbol &$q_C$\\
\hline
% Units& $Mt^{-3}$\\
% \hline
  SI Units & \si{\watt\per\square\metre}\\
  \hline
  Equation&$q_C(t) = h_C (T_C - T_W(t))$, over area $A_C$\\
  \hline
  Description & $T_C$ is the temperature of the coil (\si{\celsius}).  $T_W$ is
                the temperature of the water (\si{\celsius}).  
                The heat flux out of the coil, $q_C$
                (\si{\watt\per\square\metre}), is found by assuming that
                Newton's Law of Cooling applies (\aref{A_Newt_coil}).  This law
                (\dref{NL}) is used on the surface of the coil, which has area
                $A_C$ (\si{\square\metre}) and heat transfer coefficient $h_C$
                (\si{\watt\per\square\metre\per\celsius}).  This equation
                assumes that the temperature of the coil is constant over time
                (\aref{A_tcoil}) and that it does not vary along the length of
                the coil (\aref{A_tlcoil}).  \\
  \hline
  Sources& Citation here \\
  \hline
  Ref.\ By & \iref{ewat}\\
  \hline
\end{tabular}
\end{minipage}\\

% \subsubsection{Data Types}\label{sec_datatypes}

% \plt{This section is optional.  In many scientific computing programs it isn't
%   necessary, since the inputs and outpus are straightforward types, like
%   reals, integers, and sequences of reals and integers.  However, for some
%   problems it is very helpful to capture the type information.}

% \plt{The data types are not derived; they are simply stated and used by other
%   models.}

% \plt{All data types must be used by at least one of the models.}

% \plt{For the mathematical notation for expressing types, the recommendation is
%   to use the notation of~\cite{HoffmanAndStrooper1995}.}

% This section collects and defines all the data types needed to document the
% models. \plt{Modify the examples below for your problem, and add additional
% definitions as appropriate.}

% ~\newline

% \noindent \begin{minipage}{\textwidth} \renewcommand*{\arraystretch}{1.5}
% \begin{tabular}{| p{\colAwidth} | p{\colBwidth}|} \hline \rowcolor[gray]{0.9}
% Type Name & Name for Type\\
%   \hline Type Def & mathematical definition of the type\\
%   \hline Description & description here \\
%   \hline Sources & Citation here, if the type is borrowed from another
%   source\\
%   \hline \end{tabular} \end{minipage}\\

\subsubsection{Instance Models} \label{sec_instance}    

\plt{The motivation for this section is to reduce the problem defined in
  ``Physical System Description'' (Section~\ref{sec_phySystDescrip}) to one
  expressed in mathematical terms. The IMs are built by refining the TMs and/or
  GDs.  This section should remain abstract.  The SRS should specify the
  requirements without considering the implementation.}

This section transforms the problem defined in Section~\ref{Sec_pd} into one
which is expressed in mathematical terms. It uses concrete symbols defined in
Section~\ref{sec_datadef} to replace the abstract symbols in the models
identified in Sections~\ref{sec_theoretical} and~\ref{sec_gendef}.

The goal \gsref{G_find_index} is solved by \iref{I_find_index} and
\gsref{G_calculate_RMSE} is solved by \iref{I_RMSE}.

~\newline

%Instance Model 1

\noindent
\begin{minipage}{\textwidth}
\renewcommand*{\arraystretch}{1.5}
\begin{tabular}{| p{\colAwidth} | p{\colBwidth}|}
  \hline
  \rowcolor[gray]{0.9}
  Number& IM\refstepcounter{instnum}\theinstnum \label{I_find_index}\\
  \hline
  Label& \bf Algorithm to detect R-wave peak index\\
  \hline
  Input& $u$, $f_s$, $N$ \\
  \hline
  Output&
  \begin{equation}
    u_f[n] = -\sum_{k=1}^{M}a_ku_f[n-k] + \sum_{l=0}^{N}b_lu[n-l] \nonumber
  \end{equation}\\
  &\begin{equation} u_w[n] = \frac{1}{W}\sum_{k=0}^{W-1}u_f[n-k]^2 \nonumber
  \end{equation}\\
  &\begin{equation} u_{th}[n] = 
      \begin{cases}
        u_w[n] & \text{if } u_w[n] \geq \theta \\
        0 & \text{if } u_w[n] < \theta \end{cases} \nonumber \end{equation}\\
  &\begin{equation} R[i] = \arg\max_{n}{u_{th}[n]}\nonumber \end{equation}\\
  \hline
  Description& When calculating $u_f$, The filter parameters $a$ and $b$ here
  are linear combinations of a series of filter parameters from \dref{B-trans}.
  The sequence of filters including low-pass and high-pass filter from
  \tref{Butterworth Filter} and \tref{Chebyshev Filter} and differential filter
  from \dref{Differential filter}.  \\
  & We use the square function and sliding window integral to calculate $u_w$,
  in this way the R-wave can be amplified and the signal can be smoothed.  \\
  & \tref{Threshold Detect} is used when calculating $u_{th}$, so we can focus
  on extracting the R-wave and ignore the influence of other signals. \\
  & In the last equation, the variable $n$ has a range limitation, we only take
  a continuous period of $n$ where $u_{th}[n]$ is non-zero. \\
  \hline
  Sources& \url{https://en.wikipedia.org/wiki/Pan%E2%80%93Tompkins_algorithm} \\
  \hline
  Ref.\ By & \iref{epcm}\\
  \hline
\end{tabular}
\end{minipage}\\

~\newline

%Instance Model 2

\noindent
\begin{minipage}{\textwidth}
\renewcommand*{\arraystretch}{1.5}
\begin{tabular}{| p{\colAwidth} | p{\colBwidth}|}
  \hline
  \rowcolor[gray]{0.9}
  Number& IM\refstepcounter{instnum}\theinstnum \label{I_RMSE}\\
  \hline
  Label& \bf calculate $RMSE$ between $A$ and $R$ \\
  \hline
  Input& $A$, $R$, $L$\\
  \hline
  Output&
  \begin{equation}
    RMSE = \sqrt{\sum_{i=1}^{n}\frac{(A[i]-R[i])^2}{n}} \nonumber
  \end{equation} \\
  \hline
  Description& $A$ is the sequence of integers containing annotated index. \\
  & $R$ is the sequence of integers containing calculated index from
  \iref{I_find_index}. \\
  & $L$ is the length of $A$ and $R$.  Normally they are equal since we assumed
  \aref{A_high_quality}.  \\
  \hline
  Sources& Citation here \\
  \hline
  Ref.\ By & \iref{epcm}\\
  \hline
\end{tabular}
\end{minipage}\\

%~\newline

\subsubsection*{Derivation of ...}

\plt{The derivation shows how the IM is derived from the TMs/GDs.  In cases
  where the derivation cannot be described under the Description field, it will
  be necessary to include this subsection.}

\subsubsection{Input Data Constraints} \label{sec_DataConstraints}    

Table~\ref{TblInputVar} shows the data constraints on the input output
variables.  The column for physical constraints gives the physical limitations
on the range of values that can be taken by the variable.  The column for
software constraints restricts the range of inputs to reasonable values.  The
software constraints will be helpful in the design stage for picking suitable
algorithms.  The constraints are conservative, to give the user of the model the
flexibility to experiment with unusual situations.  The column of typical values
is intended to provide a feel for a common scenario.  The uncertainty column
provides an estimate of the confidence with which the physical quantities can be
measured.  This information would be part of the input if one were performing an
uncertainty quantification exercise.

The specification parameters in Table~\ref{TblInputVar} are listed in
Table~\ref{TblSpecParams}.

\begin{table}[!h]
  \caption{Input Variables} \label{TblInputVar}
  \renewcommand{\arraystretch}{1.2}
\noindent \begin{longtable*}{l l l l c} 
  \toprule
  \textbf{Var} & \textbf{Physical Constraints} & \textbf{Software Constraints} &
                             \textbf{Typical Value} & \textbf{Uncertainty}\\
  \midrule 
  $L$ & $L > 0$ & $L_{\text{min}} \leq L \leq L_{\text{max}}$ & 1.5
  \si[per-mode=symbol] {\metre} & 10\% \\
  \bottomrule
\end{longtable*}
\end{table}

\noindent 
\begin{description}
\item[(*)] \plt{you might need to add some notes or clarifications}
\end{description}

\begin{table}[!h]
\caption{Specification Parameter Values} \label{TblSpecParams}
\renewcommand{\arraystretch}{1.2}
\noindent \begin{longtable*}{l l} 
  \toprule
  \textbf{Var} & \textbf{Value} \\
  \midrule 
  $L_\text{min}$ & 0.1 \si{\metre}\\
  \bottomrule
\end{longtable*}
\end{table}

\subsubsection{Properties of a Correct Solution} \label{sec_CorrectSolution}

\noindent
A correct solution must exhibit \plt{fill in the details}.  \plt{These
  properties are in addition to the stated requirements.  There is no need to
  repeat the requirements here.  These additional properties may not exist for
  every problem.  Examples include conservation laws (like conservation of
  energy or mass) and known constraints on outputs, which are usually summarized
  in tabular form.  A sample table is shown in Table~\ref{TblOutputVar}}

\begin{table}[!h]
\caption{Output Variables} \label{TblOutputVar}
\renewcommand{\arraystretch}{1.2}
\noindent \begin{longtable*}{l l} 
  \toprule
  \textbf{Var} & \textbf{Physical Constraints} \\
  \midrule 
  $T_W$ & $T_\text{init} \leq T_W \leq T_C$ (by~\aref{A_charge}) \\
  \bottomrule
\end{longtable*}
\end{table}

\plt{This section is not for test cases or techniques for verification and
  validation.  Those topics will be addressed in the Verification and Validation
  plan.}

\newpage

\section{Requirements}

\plt{The requirements refine the goal statement.  They will make heavy use of
  references to the instance models.}

This section provides the functional requirements, the business tasks that the
software is expected to complete, and the nonfunctional requirements, the
qualities that the software is expected to exhibit.

\subsection{Functional Requirements}

\noindent \begin{itemize}

\item[R\refstepcounter{reqnum}\thereqnum \label{R_basic_filters}:] Implement
basic IIR and FIR filters using difference equations

\item[R\refstepcounter{reqnum}\thereqnum \label{R_Butterworth}:] Implement
parameter derivation of Butterworth filters

\item[R\refstepcounter{reqnum}\thereqnum \label{R_Chebyshev}:] Implement
parameter derivation of Chebyshev filters

\item[R\refstepcounter{reqnum}\thereqnum \label{R_squaring}:] Implement squaring
function

\item[R\refstepcounter{reqnum}\thereqnum \label{R_thresholding}:] Implement
thresholding function

\item[R\refstepcounter{reqnum}\thereqnum \label{R_RMSE}:] Implement RMSE
calculating function

\item[R\refstepcounter{reqnum}\thereqnum \label{R_R_wave_position}:] Combine
mathematical functions and filters to calculate R-wave position

\item[R\refstepcounter{reqnum}\thereqnum \label{R_cal_RMSE}:] Combine annotated
data and detected R-wave index to calculate RMSE

\end{itemize}

\plt{Every IM should map to at least one requirement, but not every requirement
  has to map to a corresponding IM.}

\subsection{Nonfunctional Requirements}

\noindent \begin{itemize}

\item[NFR\refstepcounter{nfrnum}\thenfrnum \label{NFR_Usability}:]
  \textbf{Usability} The code uses external automated tools to generate user
  manuals.

\item[NFR\refstepcounter{nfrnum}\thenfrnum \label{NFR_Maintainability}:]
  \textbf{Maintainability} The code is tested with complete verification and
  validation plan.

\item[NFR\refstepcounter{nfrnum}\thenfrnum \label{NFR_Portability}:]
  \textbf{Portability} Write cross-platform and portable code, works on at least
  Linux, Windows.

\item[NFR\refstepcounter{nfrnum}\thenfrnum \label{NFR_Reusability}:]
  \textbf{Reusability} The code is modularized.

\end{itemize}

\subsection{Rationale}

\plt{Provide a rationale for the decisions made in the documentation.  Rationale
should be provided for scope decisions, modelling decisions, assumptions and
typical values.}

\section{Likely Changes}    

\noindent \begin{itemize}

\item[LC\refstepcounter{lcnum}\thelcnum\label{LC_meaningfulLabel}:] \plt{Give
    the likely changes, with a reference to the related assumption (aref), as
    appropriate.}

\end{itemize}

\section{Unlikely Changes}    

\noindent \begin{itemize}

\item[LC\refstepcounter{lcnum}\thelcnum\label{LC_meaningfulLabel}:] \plt{Give
    the unlikely changes.  The design can assume that the changes listed will
    not occur.}

\end{itemize}

\section{Traceability Matrices and Graphs}

The purpose of the traceability matrices is to provide easy references on what
has to be additionally modified if a certain component is changed.  Every time a
component is changed, the items in the column of that component that are marked
with an ``X'' may have to be modified as well.  Table~\ref{Table:trace} shows
the dependencies of theoretical models, general definitions, data definitions,
and instance models with each other. Table~\ref{Table:R_trace} shows the
dependencies of instance models, requirements, and data constraints on each
other. Table~\ref{Table:A_trace} shows the dependencies of theoretical models,
general definitions, data definitions, instance models, and likely changes on
the assumptions.

\plt{You will have to modify these tables for your problem.}

\plt{The traceability matrix is not generally symmetric.  If GD1 uses A1, that
  means that GD1's derivation or presentation requires invocation of A1.  A1
  does not use GD1.  A1 is ``used by'' GD1.}

\plt{The traceability matrix is challenging to maintain manually.  Please do
  your best.  In the future tools (like Drasil) will make this much easier.}

\afterpage{
\begin{landscape}
\begin{table}[h!]
\centering
\begin{tabular}{|c|c|c|c|c|c|c|c|c|c|c|c|c|c|c|c|c|c|c|c|}
\hline
	& \aref{A_OnlyThermalEnergy}& \aref{A_hcoeff}& \aref{A_mixed}& \aref{A_tpcm}&
	\aref{A_const_density}& \aref{A_const_C}& \aref{A_Newt_coil}& \aref{A_tcoil}&
	\aref{A_tlcoil}& \aref{A_Newt_pcm}& \aref{A_charge}& \aref{A_InitTemp}&
	\aref{A_OpRangePCM}& \aref{A_OpRange}& \aref{A_htank}& \aref{A_int_heat}&
	\aref{A_vpcm}& \aref{A_PCM_state}& \aref{A_Pressure} \\
\hline
\tref{T_COE}        & X& & & & & & & & & & & & & & & & & & \\ \hline
\tref{T_SHE}        & & & & & & & & & & & & & & & & & & & \\ \hline
\tref{T_LHE}        & & & & & & & & & & & & & & & & & & & \\ \hline
\dref{NL}           & & X& & & & & & & & & & & & & & & & & \\ \hline
\dref{ROCT}         & & & X& X& X& X& & & & & & & & & & & & & \\ \hline
\ddref{FluxCoil}    & & & & & & & X& X& X& & & & & & & & & & \\ \hline
\ddref{FluxPCM}     & & & X& X& & & & & & X& & & & & & & & & \\ \hline
\ddref{D_HOF}       & & & & & & & & & & & & & & & & & & & \\ \hline
\ddref{D_MF}        & & & & & & & & & & & & & & & & & & & \\ \hline
\iref{ewat}         & & & & & & & & & & & X& X& & X& X& X& & & X \\ \hline
\iref{epcm}         & & & & & & & & & & & & X& X& & & X& X& X& \\ \hline
\iref{I_HWAT}       & & & & & & & & & & & & & & X& & & & & X \\ \hline
\iref{I_HPCM}       & & & & & & & & & & & & & X& & & & & X & \\ \hline
\lcref{LC_tpcm}     & & & & X& & & & & & & & & & & & & & & \\ \hline
\lcref{LC_tcoil}    & & & & & & & & X& & & & & & & & & & & \\ \hline
\lcref{LC_tlcoil}   & & & & & & & & & X& & & & & & & & & & \\ \hline
\lcref{LC_charge}   & & & & & & & & & & & X& & & & & & & & \\ \hline
\lcref{LC_InitTemp} & & & & & & & & & & & & X& & & & & & & \\ \hline
\lcref{LC_htank}    & & & & & & & & & & & & & & & X& & & & \\
\hline
\end{tabular}
\caption{Traceability Matrix Showing the Connections Between Assumptions and Other Items}
\label{Table:A_trace}
\end{table}
\end{landscape}
}

\begin{table}[h!]
\centering
\begin{tabular}{|c|c|c|c|c|c|c|c|c|c|c|c|c|c|c|c|c|c|c|c|c|c|c|c|}
\hline        
	& \tref{T_COE}& \tref{T_SHE}& \tref{T_LHE}& \dref{NL}& \dref{ROCT} &
	\ddref{FluxCoil}& \ddref{FluxPCM} & \ddref{D_HOF}& \ddref{D_MF}& \iref{ewat}&
	\iref{epcm}& \iref{I_HWAT}& \iref{I_HPCM} \\
\hline
\tref{T_COE}     & & & & & & & & & & & & & \\ \hline
\tref{T_SHE}     & & & X& & & & & & & & & & \\ \hline
\tref{T_LHE}     & & & & & & & & & & & & & \\ \hline
\dref{NL}        & & & & & & & & & & & & & \\ \hline
\dref{ROCT}      & X& & & & & & & & & & & & \\ \hline
\ddref{FluxCoil} & & & & X& & & & & & & & & \\ \hline
\ddref{FluxPCM}  & & & & X& & & & & & & & & \\ \hline
\ddref{D_HOF}    & & & & & & & & & & & & & \\ \hline
\ddref{D_MF}     & & & & & & & & X& & & & & \\ \hline
\iref{ewat}      & & & & & X& X& X& & & & X& & \\ \hline
\iref{epcm}      & & & & & X& & X& & X& X& & & X \\ \hline
\iref{I_HWAT}    & & X& & & & & & & & & & & \\ \hline
\iref{I_HPCM}    & & X& X& & & & X& X& X& & X& & \\
\hline
\end{tabular}
\caption{Traceability Matrix Showing the Connections Between Items of Different Sections}
\label{Table:trace}
\end{table}

\begin{table}[h!]
\centering
\begin{tabular}{|c|c|c|c|c|c|c|c|}
\hline
	& \iref{ewat}& \iref{epcm}& \iref{I_HWAT}& \iref{I_HPCM}&
	\ref{sec_DataConstraints}& \rref{R_RawInputs}& \rref{R_MassInputs} \\
\hline
\iref{ewat}            & & X& & & & X& X \\ \hline
\iref{epcm}            & X& & & X& & X& X \\ \hline
\iref{I_HWAT}          & & & & & & X& X \\ \hline
\iref{I_HPCM}          & & X& & & & X& X \\ \hline
\rref{R_RawInputs}     & & & & & & & \\ \hline
\rref{R_MassInputs}    & & & & & & X& \\ \hline
\rref{R_CheckInputs}   & & & & & X& & \\ \hline
\rref{R_OutputInputs}  & X& X& & & & X& X \\ \hline
\rref{R_TempWater}     & X& & & & & & \\ \hline 
\rref{R_TempPCM}       & & X& & & & & \\ \hline
\rref{R_EnergyWater}   & & & X& & & & \\ \hline
\rref{R_EnergyPCM}     & & & & X& & & \\ \hline
\rref{R_VerifyOutput}  & & & X& X& & & \\ \hline
\rref{R_timeMeltBegin} & & X& & & & & \\ \hline
\rref{R_timeMeltEnd}   & & X& & & & & \\ 
\hline
\end{tabular}
\caption{Traceability Matrix Showing the Connections Between Requirements and Instance Models}
\label{Table:R_trace}
\end{table}

The purpose of the traceability graphs is also to provide easy references on
what has to be additionally modified if a certain component is changed.  The
arrows in the graphs represent dependencies. The component at the tail of an
arrow is depended on by the component at the head of that arrow. Therefore, if a
component is changed, the components that it points to should also be changed.
Figure~\ref{Fig_ATrace} shows the dependencies of theoretical models, general
definitions, data definitions, instance models, likely changes, and assumptions
on each other. Figure~\ref{Fig_RTrace} shows the dependencies of instance
models, requirements, and data constraints on each other.

% \begin{figure}[h!] \begin{center} %\rotatebox{-90}
%     {
%       \includegraphics[width=\textwidth]{ATrace.png}
%     }
%     \caption{\label{Fig_ATrace} Traceability Matrix Showing the Connections Between Items of Different Sections}
%   \end{center} \end{figure}


% \begin{figure}[h!] \begin{center} %\rotatebox{-90}
%     {
%       \includegraphics[width=0.7\textwidth]{RTrace.png}
%     }
%     \caption{\label{Fig_RTrace} Traceability Matrix Showing the Connections Between Requirements, Instance Models, and Data Constraints}
%   \end{center} \end{figure}

\section{Development Plan}

\plt{This section is optional.  It is used to explain the plan for developing
  the software.  In particular, this section gives a list of the order in which
  the requirements will be implemented.  In the context of a course  this is
  where you can indicate which requirements will be implemented as part of the
  course, and which will be ``faked'' as future work.  This section can be
  organized as a prioritized list of requirements, or it could should the
  requirements that will be implemented for ``phase 1'', ``phase 2'', etc.}

\section{Values of Auxiliary Constants}

\plt{Show the values of the symbolic parameters introduced in the report.}

\plt{The definition of the requirements will likely call for
SYMBOLIC\_CONSTANTS.  Their values are defined in this section for easy
maintenance.}

\plt{The value of FRACTION, for the Maintainability NFR would be given here.}

\newpage

%\bibliographystyle {plainnat}
\bibliography {../../refs/References, ../../refs/bib}
\bibliographystyle{unsrt}

\newpage

\noindent \plt{The following is not part of the template, just some things to
  consider when filing in the template.}

\noindent \plt{Grammar, flow and \LaTeX advice:
\begin{itemize}
\item For Mac users \texttt{*.DS\_Store} should be in \texttt{.gitignore}
\item \LaTeX{} and formatting rules
\begin{itemize}
\item Variables are italic, everything else not, includes subscripts (link to
  document)
\begin{itemize}
\item \href{https://physics.nist.gov/cuu/pdf/typefaces.pdf}{Conventions}
\item Watch out for implied multiplication
\end{itemize}
\item Use BibTeX
\item Use cross-referencing
\end{itemize}
\item Grammar and writing rules
\begin{itemize}
\item Acronyms expanded on first usage (not just in table of acronyms)
\item ``In order to'' should be ``to''
\end{itemize}
\end{itemize}}

\noindent \plt{Advice on using the template:
\begin{itemize}
\item Difference between physical and software constraints
\item Properties of a correct solution means \emph{additional} properties, not a
  restating of the requirements (may be ``not applicable'' for your problem). If
  you have a table of output constraints, then these are properties of a correct
  solution.
\item Assumptions have to be invoked somewhere
\item ``Referenced by'' implies that there is an explicit reference
\item Think of traceability matrix, list of assumption invocations and list of
  reference by fields as automatically generatable
\item If you say the format of the output (plot, table etc), then your
  requirement could be more abstract
\end{itemize}
}

\end{document}